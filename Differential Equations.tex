\documentclass[UTF8,oneside]{ctexbook}

\title{Differential Equations}
\pagestyle{empty}
\date{}
\usepackage{graphicx}
\usepackage{subcaption}
\usepackage{amsmath,amssymb}
\usepackage{enumitem}
\usepackage{tabularx} % 自动调节列宽
\usepackage{booktabs} % 更好看的表格线
\usepackage{array} % 改善列格式


\usepackage[most]{tcolorbox} % 导入tcolorbox

% 定义一个 theorem 风格的方框
\tcbset{
  mytheorem/.style={
    enhanced,
    colback=white,          % 背景颜色
    colframe=red!70!black,  % 边框颜色
    colbacktitle=red!85!black, % 标题背景色
    coltitle=white,         % 标题文字颜色
    fonttitle=\bfseries,    % 标题加粗
    attach boxed title to top left={xshift=2mm,yshift=-2mm}, % 标题位置
    boxed title style={sharp corners}, % 标题方角
    sharp corners,           % 四角直角(不要圆角)
    top=3mm, bottom=3mm, left=3mm, right=3mm, % 内边距
  }
}




\begin{document}


\maketitle

\chapter*{1 一阶微分方程}

\section*{1.1 一阶变量分离性微分方程}

一阶变量分离性(separable)方程的标准形式为:

\[
\frac{dy}{dx} = f(x)g(y)
\]


其中 \(f(x)\) 和 \(g(y)\) 是分别仅依赖于 \(x\) 和 \(y\) 的函数。解此类方程的步骤如下。

一、当$g(y) \ne 0 $时:

\begin{enumerate}
    \item 将变量分离:将所有含 \(y\) 的项移到方程的一侧,含 \(x\) 的项移到另一侧:
\[\frac{1}{g(y)} dy = f(x) dx
\]
    \item 对两边进行积分:
\[\int \frac{1}{g(y)} dy = \int f(x) dx + C
\]

其中 \(C\) 是积分常数。
    \item 得\[ G(y) = F(x) + C \]
\end{enumerate}


二、当$g(y) = 0 $时:

设$y=y_0$是$g(y)=0$的解。

\[
\frac{dy_0}{dx} = f(x)g(y_0)
\]

左右两侧均为$0$,等式成立。因此,$y=y_0$是原方程的一个解。


\section*{1.2 一阶线性微分方程}
一阶线性(linear)微分方程的标准形式为:

\[
\frac{dy}{dx} + P(x)y = Q(x)
\]

其中 \(P(x)\) 和 \(Q(x)\) 是仅依赖于 \(x\) 的已知函数。解此类方程有两个方法。

\subsection*{1.2.1 积分因子法}
积分因子法的本质是将左边变成\textbf{乘积函数}的求导,然后再两边积分。

\begin{enumerate}
    \item 计算积分因子 \(\mu(x)\):
    
\[
\mu (x) = e^{\int P(x) \, dx}
\]
    \item 将原方程两边乘以 $\mu (x)$:
\[
\mu (x) \frac{dy}{dx} + \mu (x) P(x) y = \mu (x) Q(x)
\]

注意到左边是一个乘积函数,则

\[
\frac{d}{dx} [\mu (x) y] = \mu (x) Q(x)
\]
    \item 对两边进行积分:
\[
\mu (x) y = \int \mu (x) Q(x) \, dx + C
\]

\[
\Rightarrow y = \frac{\int \mu(x) Q(x) \, dx + C}{\mu (x)}
\]


\end{enumerate}

\subsection*{1.2.2 常数变易法}
常数变易法的本质是\textbf{非齐次方程的解=齐次方程的通解+一个非齐次方程的特解}。其中,齐次解含有常数$C$,而非齐次特解不含常数$C$。

\begin{enumerate}
    \item 先解对应的齐次方程\[
\frac{dy}{dx} + P(x)y = 0
\]
解得齐次解
\[y = Ce^{-\int P(x) \, dx}\]
    \item 设非齐次解为 $y=\mu (x)e^{-\int P(x) \, dx}$,则
\[
y^{'}=\mu^{'}(x)e^{-\int P(x) \, dx} - \mu (x)P(x)e^{- \int P(x) \, dx}
\]
    \item 将 $y$ 和 $y^{'}$ 代入原方程,化简得  
\[\mu^{'}(x) = Q(x)e^{\int P(x) \, dx}\]
再对$\mu^{'}(x)$积分,得

\[
\mu (x) = \int Q(x)e^{\int P(x) \, dx} \, dx + C
\]

    \item 将$\mu (x)$代入$y=\mu (x)e^{-\int P(x) \, dx}$,得非齐次方程的通解
\end{enumerate}




\subsection*{1.2.3 伯努利(Bernoulli)方程}

伯努利方程(Bernoulli Equations) 是一阶常微分方程

\[
\frac{dy}{dx}+P(X)y=Q(x)y^n
\]

其中,$n \ne 0,1$。方程含非线性项$y^n(n\ne 1)$,可通过变量代换转化为\textbf{一阶线性微分方程求解}。

例题:$y'-\varepsilon y= -\sigma y^{-2}$

令$z=y^{1-n}=y^3$,则$\frac{dz}{dx}=3y^2\frac{dy}{dx}$。左右两边同乘$3y^2$,得

\[
3y^2\frac{dy}{dx}-3\varepsilon y^3=-3\sigma
\]

\[
\Rightarrow \frac{dz}{dx}-3\varepsilon z=-3\sigma
\]

用积分因子法可求解,并将$z=y^3$代回得到结果。

\section*{1.3 零次齐次方程}
零次齐次方程满足:
\[
F(tx,ty) = t^0 F(x,y) = F(x,y) 
\]

\[
\Rightarrow \frac{dy}{dx} = F\left(\frac{y}{x}\right)
\]

右边仅依赖于$\frac{y}{x}$的函数,不含独立常数项。

若右边含有常数项,如
\[
\frac{dy}{dx} = \frac{2x+y+1}{x+2y+3}
\]

可通过变量代换将其化为零次齐次方程。
\begin{itemize}
    \item 令$x=X+h$,$y=Y+k$,其中$h,k$为常数
    \item 令常数项为$0$,即解方程组计算$h,k$,使右端只含$\frac{Y}{X}$的函数
    \item 解零次齐次方程$\frac{dY}{dX}=F\left(\frac{Y}{X}\right)$
    \item 将$X=x-h$,$Y=y-k$代回
    \item 若无法解方程组,则说明该方程不是零次齐次方程
\end{itemize}
 
\vspace{1\baselineskip}
解零次齐次方程的步骤如下:

\begin{enumerate}
    \item 令 $u=\frac{y}{x}$,则 $y=ux$,$\frac{dy}{dx} = u + x\frac{du}{dx}$
    \item 代入$\frac{dy}{dx} = F\left(\frac{y}{x}\right)$,得 
\[
u + x\frac{du}{dx} = F(u)
\]

\[
\frac{du}{F(u) - u} = \frac{dx}{x}
\]

    \item 两边积分
\[
\int \frac{1}{F(u) - u} \, du = \int \frac{1}{x} \, dx
\]

最后将$u=\frac{y}{x}$代回。

\end{enumerate}


\section*{1.4 线性方程与非线性方程}

一阶线性微分方程的标准形式为:
\[\frac{dy}{dx} + P(x)y = Q(x)
\]

其中:
\begin{itemize}
    \item $y$和$\frac{dy}{dx}$均以一次幂出现
    \item 系数只能是关于$x$的函数,不能含有$y$
    \item 不允许出现$y^2$,$y\cdot \frac{dy}{dx}$,$\frac{1}{y}$,$e^y$等非线性项
\end{itemize}

否则,该方程为非线性方程。

\vspace{1\baselineskip}

当方程以$\dfrac{dy}{dx} = f(x,y)$形式出现但求解困难时,可尝试将其整理为

\[
\dfrac{dx}{dy} + P(y)x = Q(y)
\]

交换自变量和因变量。

\subsection*{1.4.1 存在性与唯一性}


\begin{tcolorbox}[mytheorem,
  title=Theorem 4.1 \quad 存在性与唯一性]



无论线性方程or非线性方程,对于初值问题(Initial Value Problem,I.V.P)

\[
\begin{cases}
    y' = f(t,y)\\
    y(t_0)=y_0
\end{cases}
\]

当$f(t,y)$和$\frac{\partial f}{\partial y}$在$(t_0,y_0)$附近连续时,I.V.P存在且唯一解。


\end{tcolorbox}

其中:

\begin{itemize}
    \item $f(t,y)$连续$\rightarrow$I.V.P存在解
    \item $f(t,y)$连续 $\&$ $\frac{\partial f}{\partial y}$连续$\rightarrow$I.V.P存在且唯一解
\end{itemize}


\vspace{2\baselineskip}

\section*{1.5 自治微分方程与非自治微分方程}

\subsection*{1.5.1 自治微分方程}

所有一阶微分方程可分为自治方程和非自治方程。

自治(exact)方程的标准形式为:
\[\frac{dy}{dx} = f(y)
\]

非自治方程的标准形式为:
\[\frac{dy}{dx} = f(x,y)
\]

\subsection*{1.5.2 方向场}
对于一阶微分方程

\[
\frac{dy}{dx} = f(x,y)
\]

在平面上每一点$(x,y)$处,斜率为$f(x,y)$的线段称为该点的方向元素。所有方向元素构成的图形称为该方程的方向场。

\subsubsection{自治方程}


\textbf{平衡解}是$y$与$x$无关的解,即\textbf{常数解}。平衡解满足$\frac{dy}{dx}=0$。

自治方程方向场的图像特征:

\begin{itemize}
    \item 是否含平衡解,取决于$\frac{dy}{dx}=0$是否有解
    \item 若含平衡解$y=y_0$,则在$y=y_0$处方向元素水平
    \item 方向元素与$x$无关,只与$y$有关,表现为每个水平方向上,小斜线均重复
\end{itemize}

\begin{figure}[htbp]
    \centering
    \includegraphics[width=1.0\textwidth]{figures/1.5.2_Autonomous.png}
    \caption{自治方程方向场示例}
    \label{fig:自治方程方向场}
\end{figure}

如图\ref{fig:自治方程方向场}所示,方程$\frac{dp}{dt} = 0.5p -450$的解为$p=900+Ce^{t/2}$,方向场中平衡解$p=900$处方向元素水平。

\subsubsection{非自治方程}

目前遇到的非自治方程的例子——一阶线性非齐次方程

\[
\frac{dy}{dx} + P(x)y = Q(x)
\]

其中 $Q(x) \ne 0$。

解的结构为:齐次解(含$C$)+特解(不含$C$)。

非自治方程方向场的图像特征:

\begin{itemize}
    \item 不含平衡解
    \item 方向元素与$x$有关,每个水平方向上,小斜线不重复
    \item 随着$x \to \pm \infty$,所有解趋向于/远离特解
\end{itemize}

\begin{figure}[htbp]
    \centering
    \includegraphics[width=1.0\textwidth]{figures/1.5.2_Non_Autonomous.png}
    \caption{非自治方程方向场示例}
    \label{fig:非自治方程方向场}
\end{figure}

如图\ref{fig:非自治方程方向场}所示,方程$\frac{dy}{dt} + \frac{1}{2}y = \frac{1}{2}e^{t/3}$的解为$y=\frac{3}{5}e^{t/3}+Ce^{-t/2}$,方向场中无平衡解。

但随着$t \to + \infty$,所有解趋向于特解$y=\frac{3}{5}e^{t/3}$。

\vspace{2\baselineskip}

\subsection*{1.5.3 逻辑斯蒂函数}

逻辑斯蒂函数(Logistic)用于描述具有饱和增长特性的种群增长


\[
\frac{dy}{dt} = r(1-\frac{y}{K})y
\]

其中$r$是增长率,$K$是环境容纳量。解该微分方程,得

\[
y = \frac{K}{1+Ce^{-rt}}
\]

若初始值$y(0)=y_0$,则

\[
y =\frac{y_0 K}{y_o+(K-y_0)e^{-rt}}
\]

\vspace{5\baselineskip}

\subsubsection{平衡解}

令$y'=0$,得平衡解$y=0$和$y=K$。

\begin{itemize}
    \item 当$0<y<K$,$\frac{dy}{dt}>0$,$y(t)$单调递增
    \item 当$y>K$,$\frac{dy}{dt}<0$,$y(t)$单调递减
\end{itemize}



$\therefore$  当$ t \rightarrow \infty,~ y = \frac{K}{1+Ce^{-rt}} \rightarrow K$


即解趋近于$K$。

\vspace{1\baselineskip}

稳定的平衡解指:系统暂时偏离这个平衡解,但最终仍能回到平衡解。

\begin{itemize}
    \item 对于$y=0$:当$y$略大于$0$时,$\frac{dy}{dt} > 0$,即$y$远离平y衡解$0$。$\therefore y=0$不是稳定的平衡解。
    \item 对于$y=K$:当$y$略大于$K$时,$\frac{dy}{dt} < 0$,即$y$靠近平衡解$K$;当$y$略小于$K$时,$\frac{dy}{dt} > 0$,即$y$靠近平衡解$K$。$\therefore y=K$是稳定的平衡解。
\end{itemize}




\subsubsection{凹凸性与拐点}

计算$y$的二阶导

\[
\frac{d^2y}{dt^2} = r^2(1-\frac{y}{K})y(1-\frac{2y}{K})
\]

随着$t \rightarrow\infty$,最终$y\rightarrow \infty$。故$0<y<K$。

\begin{itemize}
    \item 当$0<y<\frac{K}{2},\frac{d^y}{dt^2} > 0$,曲线是凹的。
    \item 当$\frac{K}{2}<y<K,\frac{d^y}{dt^2} < 0$,曲线是凸的。
\end{itemize}

拐点处$\frac{d^y}{dt^2}=0$,故拐点为$y=\frac{K}{2}$。

\vspace{7\baselineskip}

\section*{1.6 恰当方程与积分因子}


\subsection*{1.6.1 恰当方程}
\begin{tcolorbox}[mytheorem,
    title=Theorem 1.6.1 \quad 恰当性的判定定理]

对于方程

\[
M(x,y) \, dx+N(x,y) \, dy=0
\]
    
若存在二元函数$u(x,y)$,使$\frac{\partial u}{\partial x}=M(x,y),\frac{\partial u}{\partial y}=N(x,y)$,则方程$M \, dx+N \, dy=0$称为恰当方程。
且充要条件为

\[
\frac{\partial M}{\partial y}=\frac{\partial N}{\partial x}
\]

\end{tcolorbox}

例题:$(2x+y) \, dx+(x+2y) \, dy=0$

设$M=2x+y,N=x+2y.$则 $\frac{\partial M}{\partial y}=1=\frac{\partial N}{\partial x}$,符合恰当方程的定义。




\[
\frac{\partial u}{\partial x}=2x+y  \Rightarrow u = \int (2x+y) \, dx = x^2+xy+ \underset{\text{常数可能与 $y$ 有关}}{\varphi(y)}
\]

\[
\Rightarrow \frac{\partial u}{\partial y}=x+\varphi ' (y) =N = x+2y
\]

$\therefore \varphi '(y)=2y,\varphi(y)=y^2+C$。

$\therefore u(x,y)=x^2+xy+y^2$,通解为$x^2+xy+y^2=C$。

\subsection*{1.6.2 积分因子}

设积分因子$v(x,y)$,使$v(x,y)M \, dx+v(x,y) N \, dy =0$为恰当方程。
则需满足:

\[
\frac{\partial(vM)}{\partial y} = \frac{\partial (vN)}{\partial x} \Rightarrow Mv_y-Nv_x+(M_y-N_x) v=0
\]

\begin{itemize}
    \item 当$v$仅依赖$x$,则满足$\frac{dv}{dx}=\frac{M_y-N_x}{N}v$,此时$v=v(x)$
    \item 当$v$仅依赖$y$,则满足$\frac{dv}{dy}=\frac{M_y-N_x}{-M}v$,此时$v=v(y)$
\end{itemize}

\section*{1.7 数值逼近:欧拉方法}

欧拉方法是求解一阶初值问题的基础数值近似方法。

\[
\begin{cases}
    \text{D.E.} \quad \dfrac{dy}{dt} = f(t,y) \\
    \text{I.C.} \quad y(0) = y_0
\end{cases}
\]

\paragraph{1. 局部切线近似:}

\quad

解函数$y=\phi(t)$在$t=t_0$处的切线斜率为$\dfrac{dy}{dt} = f(t_0,y_0)$,在$t_0$附近,用切线方程近似解曲线

\[
y_1 = y_0 + f(t_0,y_0)(t_1-t_0)
\]

\paragraph{2. 迭代扩展}

取步长$h = t_{n+1} - t_n$(即每一步的间隔相同),得迭代公式为

\[
y_{n+1} = y_n + h \cdot f(t_n,y_n), \quad n = 0, 1, 2, \cdots
\]

欧拉方法易实现,但精度有限:

\begin{itemize}
    \item 步长$h$越小,精度越高,但计算量越大
    \item 收敛解族的误差增长缓慢,但发散解族的误差会急剧放大,需采用更高效的数值方法
\end{itemize}

\vspace{14\baselineskip}


\section*{1.8 解的存在唯一性定理}

\subsection*{1.8.1 解的存在唯一性定理}

\begin{tcolorbox}[mytheorem,
    title=Theorem 1.8.1 \quad 一阶微分方程的存在唯一性定理]

考虑初值问题

\[
\begin{cases}
    y' = f(t,y)\\
    y(0) = 0
\end{cases}
\]
    
若f(t,y)在矩形区域$R:|t| \le a,|y| \le b$上连续,则该初值问题在该区域内存在唯一解。

\end{tcolorbox}

\subsection*{1.8.2 微分方程与积分方程的等价性}

对于初值问题

\[
\begin{cases}
    y' = f(t,y)\\
    y(t_0) = y_0
\end{cases}
\]

对$y'=f(t,y)$从$t_0$到$t$积分,并结合初始条件$y(t_0) = y_0$,可得积分方程

\[
\phi (t) = y_0 + \int_{t_0}^{t}f(s,\phi(s)) \, ds
\]

即:微分方程初值问题的解$\phi(t)$与该积分方程的解完全等价。

\subsection*{1.8.3 皮卡逐步逼近法}

对于初值问题

\[
\begin{cases}
    y' = f(t,y)\\
    y(t_0) = y_0
\end{cases}
\]

定义常数函数$\phi_0(t_0) = y_0$,有第n次近似解

\[
\phi_{n+1}(t) = y_0 + \int_{t_0}^t f(s,\phi_n(s)) \, ds
\]

\subsection*{1.8.4 里卡蒂(Riccati)方程}

Riccati方程是一阶非线性常微分方程,其标准形式为

\[
\dfrac{dy}{dt} = q_1(t)+q_2(t)y+q_3(t)y^2
\]

需已知某特解$y_1(t)$,令$y(t) = z(t) + y_1(t)$,则

\[
\dfrac{dy}{dt} = \frac{dz}{dt} + \frac{dy_1}{dt}
\]

将其代入原方程

\[
\begin{aligned}
    \frac{dz}{dt}+\frac{dy_1}{dt}
    &= q_1(t)+q_2(t)(z+y_1)+q_3(t)(z+y_1)^2 \\
    &= \left[q_1(t)+q_2(t)y_1+q_3(t)y_1^2\right]
    + q_2(t)z + q_3(t)z^2 + 2q_3(t)y_1 z
\end{aligned}
\]

由于$\dfrac{dy}{dt} = q_1(t)+q_2(t)y_1+q_3(t)y_1^2$,上式可简化

\[
\frac{dz}{dt} = (q_2(t) + 2q_3(t)y_1(t))z+q_3(t)z^2
\]

化简为Bernoulli方程,作变量替换$u=z^{-1}$可解。

\chapter*{2 二阶微分方程}

\section*{2.1 可降阶的二阶微分方程}

降阶法科求解部分\text{非线性}、\text{变系数}的二阶方程。

\subsection*{2.1.1 $y''=f(x)$}

仅含$x$。两次积分即可求解。

\begin{itemize}
    \item 第一次积分:$y'=\int f(x) \, dx + C_1$
    \item 第二次积分:$y=\int (\int f(x) \, dx + C_1) \, dx +C_2$
\end{itemize}

\subsection*{2.1.2 $y''=f(x,y')$}

不含$y$。令$p=y'$,则$y''=\dfrac{dp}{dx}$,方程变为一阶方程$\dfrac{dp}{dx}f(x,p)$。

\subsection*{2.1.3 $y''=f(y,y')$}

不含$x$。令$p=y'$,则$y''=p\dfrac{dp}{dy}$,方程变为一阶方程$p\dfrac{dp}{dy}=f(y,p)$。

\vspace{5\baselineskip}

\section*{2.2 二阶线性常系数微分方程}

对于二阶线性常系数齐次方程,有以下定理

\[
y'' + py' + qy = 0
\]

\subsubsection*{1. 微分算子与存在唯一性定理}

\paragraph{微分算子}

\quad

设$p(x),q(x)$是区间$I$上的连续函数,定义微分算子

\[
L[\phi] = \phi''+p(x)\phi'+q(x)\phi
\]

它将函数$\phi_(x)$映射为另一个函数。

\paragraph{存在唯一性定理}

\quad

对于初值问题

\[
\begin{cases}
    y''+p(x)y'+q(t)y = f(x)\\
    y(x_0) = y_0,y'(x_0) = y_0'
\end{cases}
\]

若$p,q,f$在包含$t_0$的开区间$I$上连续,则该问题在$I$上存在唯一解。

\subsubsection*{2. 叠加原理}

若$y_1,y_2$是$L[y] = 0$的解,则对任意常数$c_1,c_2$,线性组合$y=c_1y_1+c_2y_2$也是解。

\subsubsection*{3. 朗斯基行列式(Wronskian)与基本解组}

对两个解$y_1,y_2$,定义朗斯基行列式

\[
W[y_1,y_2](t) = 
\begin{vmatrix}
y_1(t)&y_2(t)\\
y_1'(t)&y_2(t)
\end{vmatrix}
\]

若$W[y_1,y_2](t)\ne 0$,则$y_1,y_2$线性无关,构成基本解组,方程的通解为

\[
y=c_1y_1+c_2y_2
\]

\subsubsection*{4. 复值解的实部与虚部}

若$y=u(t)+iv(t)$是方程$L[y]=0$的复值解,则其实部$u(t)$和虚部$v(t)$也是方程的解。

\subsubsection*{5. 阿贝尔定理(Abel's Theorem)}

若$y_1,y_2$是方程$y''+p(t)y'+q(t)y=0$的解,则朗斯基行列式满足

\[
W[y_1,y_2](t) = ce^{-\int p(t) \, dt} 
\]

其中$c=W(t_0)$

\subsubsection*{6. 克莱姆法则(Cramer)}

设二阶线性齐次微分方程的通解为$y(t) = c_1y_1(t) +c_2y_2(t)$,给定初值条件$y(t_0)=y_0,y'(t_0)=y_0'$,需要确定习俗$c_1,c_2$。将初值代入通解

\[
\begin{cases}
    c_1y_1(t_0)+c_2y_2(t_0)=y_0\\
    c_1y_1'(t_0)+c_2y_2'(t_0)=y_0'
\end{cases}
\]

\paragraph{求$c_1$:}

\quad

将系数矩阵的第一列替换为初值向量
$\begin{bmatrix}
    y_0\\y_0'
\end{bmatrix}$,则

\[
c_1=\dfrac{\begin{vmatrix}y_0&y_2(t_0)\\y_0'&y_2'(t_0)\end{vmatrix}}{W[y_1,y_2](t_0)}
\]

\paragraph{求$c_2$:}

\quad

将系数矩阵的第二列替换为初值向量
$\begin{bmatrix}
    y_0\\y_0'
\end{bmatrix}$,则

\[
c_2=\dfrac{\begin{vmatrix}y_1(t_0)&y_0\\y_1'(t_0)&y_0'\end{vmatrix}}{W[y_1,y_2](t_0)}
\]

利用朗斯基行列式判断基本解组的线性无关性,再用克莱姆法则,在已知初值条件时确定通解中的系数$c_1,c_2$。

\subsection*{恰当方程与伴随方程}


\subsection*{2.2.1 齐次方程}

考虑齐次方程

\[
y''+py'+qy=0
\]

\begin{enumerate}[label=STEP\arabic*]
    \item 写特征方程:$r^2+pr+q=0$
    \item 解特征根,按根的类型定通解:
        \begin{itemize}
            \item $r_1 \ne r_2$:$y=C_1e^{r_1x} + C_2e^{r_2x}$
            \item $r_1=r_2=r$:$=C_1e^{rx}+C_2xe^{rx}$
            \item 共轭复根$r=\alpha \pm i \beta$:$y=e^{\alpha x}(C_1 \cos \beta x + C_2 \sin \beta x)$
        \end{itemize}
\end{enumerate}

\subsection*{2.2.2 非齐次方程}

考虑非齐次方程

\[
y''+py'+qy=f(x) \; (f(x)\not\equiv 0)
\]

通解=齐次通解+非齐次通解。

用待定系数法设特解形式,该方法仅适用于$f(x)$是多项式、指数函数、正余弦函数,或它们的乘积。

\subsubsection*{1. $f(x)=P_m(x)$($m$次多项式,如$f(x)=3x^2+2x+1$)}

\begin{table}[htbp]
\centering
\begin{tabular}{c c}
\toprule
特征根与$r=0$的关系 & 特解$y_p$假设形式  \\
\midrule
$r=0$不是特征根 & $y_p=Q_m(x)=Ax^2+Bx+C$  \\
$r=0$是单特征根 & $y_p=x\cdot Q_m(x)$  \\
$r=0$是二重特征根 & $y_p=x^2\cdot Q_m(x)$  \\
\bottomrule
\end{tabular}
\end{table}

\vspace{4\baselineskip}

\subsubsection*{2. $f(x)=Ae^{\alpha x}$(或$Pn(x)e^{\alpha x}$,多项式×指数的本质一致}

\begin{table}[htbp]
\centering
\begin{tabular}{c c}
\toprule
特征根与$\alpha$的关系 & 特解$y_p$假设形式  \\
\midrule
$\alpha$不是特征根 & $y_p=Be^{\alpha x}$  \\
$\alpha$是单特征根 & $y_p=x\cdot Be^{\alpha x}$  \\
$\alpha$是二重特征根 & $y_p=x^2\cdot Be^{\alpha x}$  \\
\bottomrule
\end{tabular}
\end{table}

\subsubsection*{3. $f(x) = A\cos (\beta x)+B\sin (\beta x)$(或$P_n(x)[A\cos (\beta x)+B\sin (\beta x)]$,多项式×三角函数的本质一致}

\begin{table}[htbp]
\centering
\begin{tabular}{c c}
\toprule
特征根与$i\beta$的关系 & 特解$y_p$假设形式  \\
\midrule
$i\beta$不是特征根 & $y_p=C\cos (\beta x)+D\sin (\beta x)$  \\
$i\beta$是特征根 & $y_p=x\cdot C\cos (\beta x)+D\sin (\beta x)$  \\
\bottomrule
\end{tabular}
\end{table}

\subsubsection*{4. $f(x)=e^{\alpha x}(A\cos (\beta x)+B\sin (\beta x))$(指数×正余弦)}

\begin{table}[htbp]
\centering
\begin{tabular}{c c}
\toprule
特征根与$\alpha \pm i\beta$的关系 & 特解$y_p$假设形式  \\
\midrule
$\alpha \pm i\beta$不是特征根 & $y_p=e^{\alpha x}(C\cos (\beta x)+D\sin (\beta x))$  \\
$\alpha \pm i\beta$是特征根 & $y_p=x\cdot e^{\alpha x}(C\cos (\beta x)+D\sin (\beta x))$  \\
\bottomrule
\end{tabular}
\end{table}


\section*{2.3 二阶线性变系数微分方程}

欧拉方程是典型的二阶线性变系数微分方程

\[
t^2y''(t)+\alpha t y'(t)+\beta y(t) = 0
\]

设$t=e^x$,将$y$对$t$的导数转化为对$x$的导数

\[
\dfrac{d^2y}{dx^2}+(\alpha-1)\dfrac{dy}{dx}+\beta y = 0
\]




\chapter*{3 高阶微分方程}

\section*{3.1 高阶微分方程解的理论}

对于 $n$ 阶线性微分方程,其标准形式为:
$$ L[y] = y^{(n)} + p_1(t)y^{(n-1)} + \cdots + p_n(t)y = g(t) $$
如果 $g(t) = 0$,称为齐次方程;如果 $g(t) \neq 0$,称为非齐次方程。

\subsection*{1. 朗斯基行列式 (Wronskian) 与线性无关}
对于定义在区间 $I$ 上的 $n$ 个解 $y_1, y_2, \cdots, y_n$,其线性无关性由朗斯基行列式判定:
$$ W(y_1, \cdots, y_n)(t) = \begin{vmatrix} 
y_1 & y_2 & \cdots & y_n \\
y_1' & y_2' & \cdots & y_n' \\
\vdots & \vdots & \ddots & \vdots \\
y_1^{(n-1)} & y_2^{(n-1)} & \cdots & y_n^{(n-1)}
\end{vmatrix} $$
若 $W(t_0) \neq 0$(哪怕仅在一点),则这组解线性无关,构成方程的\textbf{基础解系}。

\subsection*{2. 通解结构定理}
非齐次线性微分方程的通解结构为:
$$ y(t) = y_h(t) + y_p(t) $$
其中:
\begin{itemize}
    \item $y_h(t) = c_1y_1 + \cdots + c_ny_n$ 是对应齐次方程的通解(Complementary Solution)。
    \item $y_p(t)$ 是非齐次方程的任意一个特解(Particular Solution)。
\end{itemize}

\subsection*{典型例题 1:验证基础解系}
验证 $y_1=1, y_2=\cos t, y_3=\sin t$ 是方程 $y''' + y' = 0$ 的基础解系。

\textbf{解:}
首先代入验证:
$y_1'=0 \implies$ 满足。
$y_2''' + y_2' = \sin t - \sin t = 0 \implies$ 满足。
同理 $y_3$ 满足。
计算朗斯基行列式:
$$ W = \begin{vmatrix} 
1 & \cos t & \sin t \\
0 & -\sin t & \cos t \\
0 & -\cos t & -\sin t 
\end{vmatrix} = 1 \cdot (\sin^2 t + \cos^2 t) = 1 \neq 0 $$
因为 $W \neq 0$,故三者线性无关,构成基础解系。

\section*{3.2 常系数齐次线性微分方程}

对n阶方程$L[y]=a_0y^{(n)}+a_1y^{(n-1)}+\cdots+a_n y=0$,假设解为$y=e^{rt}$,代入得特征多项式

\[
Z(r) = a_0r^n+a_1r^{n-1}+\cdots+a_n
\]

\subsection*{1. 实根且互不相等}

若有n个不同实根$r_1,r_2,\cdots,r_n$,则通解为

\[
y=c_1e^{r_1t}+c_2e^{r_2t}+\cdots+c_ne^{r_n t}
\]

\subsection*{2. 复共轭根}

若有复共轭根$r=\lambda\pm i \mu$。则对应是实值解为$e^{\lambda t}\cos \mu t,e^{\lambda t} \sin \mu t$。

\subsection*{3. 重根}

若$r$是k重实根,对应解为$e^{rt},te^{rt},\cdots,t^{k-1}e^{rt}$。

\section*{3.3 待定系数法}

当非齐次项 $g(t)$ 为多项式、指数函数、正弦/余弦函数或其乘积时,我们可以推测特解 $y_p$ 的形式。核心原则是:$y_p$ 的形式应与 $g(t)$ 及其导数的形式一致。

\subsection*{1. 特解形式设定表}
设 $P_n(t)$ 为 $n$ 次多项式:
\begin{center}
\begin{tabular}{|c|c|}
\hline
\textbf{非齐次项 $g(t)$} & \textbf{特解试探形式 $y_p(t)$} \\
\hline
$P_n(t) = a_n t^n + \cdots + a_0$ & $t^s (A_n t^n + \cdots + A_0)$ \\
\hline
$P_n(t)e^{\alpha t}$ & $t^s (A_n t^n + \cdots + A_0) e^{\alpha t}$ \\
\hline
$P_n(t)e^{\alpha t} \begin{cases} \sin \beta t \\ \cos \beta t \end{cases}$ & $t^s e^{\alpha t} [(A_n t^n + \cdots) \cos \beta t + (B_n t^n + \cdots) \sin \beta t]$ \\
\hline
\end{tabular}
\end{center}

\subsection*{2. 修正因子 $t^s$ 的确定规则 (重难点)}
这是待定系数法中最容易出错的一步。
\textbf{规则:} 若推测的特解形式中,某一项已经是齐次方程通解 $y_h$ 中的解,则必须乘以 $t^s$,其中 $s$ 是使得该项不再是 $y_h$ 解的最小正整数(即重根次数)。
\begin{itemize}
    \item 若 $\alpha$ 不是特征根,则 $s=0$。
    \item 若 $\alpha$ 是特征方程的 $k$ 重根,则 $s=k$。
    \item 对于复数情形,看 $\alpha \pm i\beta$ 是否为特征根。
\end{itemize}

\subsection*{典型例题 2:共振情形(Correction Factor)}
求微分方程的通解:
$$ y'' - 4y' + 4y = 2e^{2t} + 8t - 12 $$

\textbf{解:}
\textbf{第一步:求齐次通解 $y_h$}
特征方程 $r^2 - 4r + 4 = 0 \implies (r-2)^2 = 0$。
特征根为 $r_1 = r_2 = 2$(二重根)。
$$ y_h(t) = c_1 e^{2t} + c_2 t e^{2t} $$

\textbf{第二步:设定特解 $y_p$ 形式}
利用叠加原理,将 $g(t)$ 分为两部分:$g_1(t) = 2e^{2t}$ 和 $g_2(t) = 8t - 12$。

1. 对于 $2e^{2t}$:
初步设想为 $A e^{2t}$。但观察发现,$e^{2t}$ 和 $te^{2t}$ 均已出现在 $y_h$ 中(因为 $r=2$ 是二重根)。
故需修正,乘以 $t^2$:
$$ y_{p1} = A t^2 e^{2t} $$

2. 对于 $8t - 12$:
这是1次多项式,且 $r=0$ 不是特征根,故无需修正:
$$ y_{p2} = Bt + C $$

综上,特解形式为 $y_p = A t^2 e^{2t} + Bt + C$。

\textbf{第三步:代入求解系数}
计算导数:
$y_p' = A(2te^{2t} + 2t^2e^{2t}) + B$
$y_p'' = A(2e^{2t} + 4te^{2t} + 4te^{2t} + 4t^2e^{2t}) = A(4t^2 + 8t + 2)e^{2t}$

代入原方程 $y'' - 4y' + 4y$:
$$ \underbrace{A(4t^2+8t+2)e^{2t}}_{y''} - 4\underbrace{[A(2t+2t^2)e^{2t}+B]}_{y'} + 4\underbrace{[At^2e^{2t}+Bt+C]}_{y} = 2e^{2t} + 8t - 12 $$

整理 $e^{2t}$ 项:
$$ At^2(4 - 8 + 4) + At(8 - 8) + 2A = 2A e^{2t} $$
(注意:$t^2$ 和 $t$ 的系数必然抵消为0,这是正确设定 $s$ 的结果)
令 $2A = 2 \implies A = 1$。

整理多项式项:
$$ 4Bt + (4C - 4B) = 8t - 12 $$
比较系数:
$4B = 8 \implies B = 2$
$4C - 4(2) = -12 \implies 4C = -4 \implies C = -1$

\textbf{第四步:写出通解}
$$ \boxed{y(t) = c_1 e^{2t} + c_2 t e^{2t} + t^2 e^{2t} + 2t - 1} $$

\section*{3.4 常数变易法}

设齐次方程的n个线性无关解为$y_1,y_2,\cdots,y_n$,则齐次通解为

\[
y_h=c_1y_1+c_2y_2+\cdots+c_ny_n
\]

将常数$c_1,\cdots,c_n$替换为待定函数$u_1(t),\cdots,u_n(t)$,假设特解

\[
y_p=u_1y_1+u_2y_2+\cdots+u_ny_n
\]

\[
W=
\begin{vmatrix}
    y_1&y_2&\cdots&y_n\\
    y_1'&y_2'&\cdots&y_n'\\
    \vdots&\vdots&\ddots&\vdots\\
    y_1^{(n-1)}&y_2^{(n-1)}&\cdots&y_n^{(n-1)}
\end{vmatrix}
\]

替换行列式$W_m$:将$W$的第$m$列替换为
$\begin{bmatrix}
    0&0&\cdots&g(t)
\end{bmatrix}^T$(共n行)

则$u_m'=\dfrac{W_m}{W}$。

\chapter*{4 二阶线性方程的幂级数解}

\section*{4.1 弗罗贝尼乌斯 (Frobenius) 方法核心笔记}

本笔记旨在总结二阶线性变系数微分方程在正则奇点处的级数解法,特别是如何快速构建指标方程与递推公式。

\subsection*{1. 正则奇点的判断与方法适用性}

对于二阶微分方程:
\begin{equation}
A(x)y'' + B(x)y' + C(x)y = 0
\end{equation}
首先将其化为标准形式 $y'' + P(x)y' + Q(x)y = 0$,其中 $P(x) = \frac{B(x)}{A(x)}, Q(x) = \frac{C(x)}{A(x)}$。

\begin{itemize}
    \item \textbf{常点 (Ordinary Point)}: 如果 $P(x)$ 和 $Q(x)$ 在 $x=x_0$ 处解析(分母不为0),则 $x_0$ 为常点。
    \item \textbf{奇点 (Singular Point)}: 如果 $P(x)$ 或 $Q(x)$ 在 $x=x_0$ 处分母为0,则 $x_0$ 为奇点。
\end{itemize}

\textbf{判定正则性 (Regularity Test)}:
计算以下两个极限(假设 $x_0=0$):
\[
p_0 = \lim_{x \to 0} x P(x), \quad q_0 = \lim_{x \to 0} x^2 Q(x)
\]
\begin{enumerate}
    \item 如果 $p_0$ 和 $q_0$ \textbf{均为有限值},则 $x=0$ 是\textbf{正则奇点}。
    \item 如果其中任意一个极限不存在或为无穷大,则为\textbf{非正则奇点}。
\end{enumerate}

\textbf{结论}:只有判定为\textbf{正则奇点}时,才可以使用弗罗贝尼乌斯方法,设解的形式为 $y = x^r \sum_{n=0}^{\infty} a_n x^n$。

\subsection*{2. 指标方程的快速求解}

无需将级数代入原方程,可利用极限值直接写出指标方程。

\textbf{快速公式}:
\begin{equation}
\boxed{r(r-1) + p_0 r + q_0 = 0}
\end{equation}

\textbf{操作步骤}:
\begin{enumerate}
    \item 将方程两端同乘 $x^2$(或 $x$),整理为以下形式(欧拉方程形式):
    \[ x^2 y'' + x[xp(x)]y' + [x^2q(x)]y = 0 \]
    确保 $y''$ 前的系数仅为 $x^2$。
    \item 读取 $x y'$ 的系数常数项即为 $p_0$。
    \item 读取 $y$ 的系数常数项即为 $q_0$。
    \item 代入公式解出 $r_1, r_2$。
\end{enumerate}

\subsection*{3. 递推公式的构建(半自动法)}

这是求解 $a_n$ 的核心技巧。我们将方程中的项分为“标准项”和“移位项(捣乱项)”。

\subsubsection{基本原理}
递推公式的结构总是:
\begin{equation}
I(n+r)a_n + \sum (\text{移位项贡献}) = 0
\end{equation}
其中 $I(r) = r(r-1) + p_0r + q_0$ 是指标多项式。

\subsubsection{如何识别与处理}

\textbf{1. 标准项 (归入 $a_n$)}:
保持 $x$ 幂次平衡的项,它们的和即为 $I(n+r)a_n$。
\begin{itemize}
    \item $x^2 y''$
    \item $x y'$
    \item 常数 $\cdot y$
\end{itemize}

\textbf{2. 移位项 (归入 $a_{n-k}$)}:
比标准项\textbf{多乘了 $x^k$} 的项。处理规则如下表:

\begin{table}[h]
\centering
\renewcommand{\arraystretch}{1.5}
\begin{tabular}{|c|c|c|c|}
\hline
\textbf{项的形式} & \textbf{多出的 $x$ 次数 ($k$)} & \textbf{对应系数归属} & \textbf{系数写法 (变标)} \\
\hline
$C \cdot x y$ & $k=1$ & $a_{n-1}$ & $C$ \\
\hline
$C \cdot x^2 y$ & $k=2$ & $a_{n-2}$ & $C$ \\
\hline
$C \cdot x^2 y'$ & $k=1$ & $a_{n-1}$ & $C(n-1+r)$ \\
\hline
$C \cdot x^3 y''$ & $k=1$ & $a_{n-1}$ & $C(n-1+r)(n-1+r-1)$ \\
\hline
\end{tabular}
\caption{移位项系数处理法则}
\end{table}

\textbf{核心口诀}:
\begin{quote}
多乘 $k$ 个 $x$,下标退 $k$ 步 ($a_{n-k}$),系数里的 $n$ 也要换成 $n-k$。
\end{quote}

\subsubsection{示例}
对于方程 $2x^2 y'' + 7x(x+1)y' - 3y = 0$:
\begin{enumerate}
    \item \textbf{找 $I(r)$}:由 $2x^2 y'' + 7xy' - 3y$ 得到 $I(r) = 2r(r-1)+7r-3$。
    \item \textbf{找移位项}:展开 $7x(x+1)y'$ 得到多余项 $7x^2 y'$。
    \item \textbf{判断移位}:$7x^2 y'$ 比 $xy'$ 多一个 $x$,故对应 $a_{n-1}$。系数由 $7(n+r)$ 变为 $7(n-1+r)$。
    \item \textbf{合成公式}:
    \[ I(n+r)a_n + 7(n+r-1)a_{n-1} = 0 \]
\end{enumerate}

\subsection*{4. 完整解题流程清单}

\begin{enumerate}
    \item \textbf{标准化}:整理方程,确保 $x=0$ 为正则奇点。
    \item \textbf{定指标}:使用快速公式求出 $r_1, r_2$。
    \item \textbf{判差值}:计算 $\Delta = r_1 - r_2$。
    \begin{itemize}
        \item $\Delta$ 非整数:大吉,直接算两个解。
        \item $\Delta$ 为整数或 0:注意小根可能含 $\ln x$,优先算大根。
    \end{itemize}
    \item \textbf{列递推}:使用半自动法写出 $a_n$ 与 $a_{n-k}$ 的关系。
    \item \textbf{代入算}:
        \begin{itemize}
            \item 将 $r_1$(大根)代入递推式,求出 $a_n$ 通项或前几项,得到 $y_1$。
            \item 若需要,对 $r_2$ 重复此步骤(仅当 $\Delta$ 非整数时安全)。
        \end{itemize}
    \item \textbf{写通解}:$y = C_1 y_1 + C_2 y_2$。
\end{enumerate}

\chapter*{5 拉普拉斯变换}

\section*{5.1 拉普拉斯逆变换与代数技巧:部分分式展开}

在求解微分方程时,我们将问题从 $t$ 域转换到 $s$ 域,得到 $Y(s)$ 后,核心难点往往在于如何求逆变换 $\mathcal{L}^{-1}\{Y(s)\}$。当 $Y(s)$ 为有理分式 $\frac{P(s)}{Q(s)}$ 时,部分分式分解(Partial Fraction Decomposition)是必不可少的工具。

\subsection*{5.1.1 不仅是简单的分解:处理既约二次因式}
考试中常出现的难点在于分母包含无法实数分解的二次项(即判别式 $\Delta < 0$)以及高次重复项。此时需要结合\textbf{配方法}与\textbf{频移定理}。

\begin{itemize}
    \item \textbf{实根情形}:对于 $(s-a)$,对应 $Ae^{at}$。
    \item \textbf{复根情形}:对于 $(s-\alpha)^2 + \beta^2$,需凑成 $\frac{s-\alpha}{(s-\alpha)^2+\beta^2}$ (余弦项) 或 $\frac{\beta}{(s-\alpha)^2+\beta^2}$ (正弦项)。
\end{itemize}

\subsubsection*{典型例题 1}
求下列函数的拉普拉斯逆变换:
$$ F(s) = \frac{3s + 5}{(s+1)(s^2 + 2s + 5)} $$

\textbf{解:}
观察分母,$(s^2+2s+5)$ 判别式小于0,应配方为 $(s+1)^2 + 4$。设部分分式形式为:
$$ \frac{3s + 5}{(s+1)((s+1)^2 + 4)} = \frac{A}{s+1} + \frac{B(s+1) + C}{(s+1)^2 + 4} $$
注意:第二项分子写成 $B(s+1)+C$ 而非 $Bs+C$ 是为了利用第一平移定理,简化后续计算。

通分并比较分子:
$$ 3s + 5 = A[(s+1)^2 + 4] + [B(s+1) + C](s+1) $$
令 $s = -1$,得 $2 = A(4) \implies A = \frac{1}{2}$。
比较 $s^2$ 系数:$0 = A + B \implies B = -A = -\frac{1}{2}$。
比较常数项:$5 = 5A + C \implies 5 = \frac{5}{2} + C \implies C = \frac{5}{2}$。

代回原式:
$$ F(s) = \frac{1}{2} \cdot \frac{1}{s+1} - \frac{1}{2} \cdot \frac{s+1}{(s+1)^2 + 2^2} + \frac{5}{4} \cdot \frac{2}{(s+1)^2 + 2^2} $$
利用逆变换线性性质及第一平移定理 $\mathcal{L}^{-1}\{F(s+a)\} = e^{-at}f(t)$:
$$ \boxed{f(t) = \frac{1}{2}e^{-t} - \frac{1}{2}e^{-t}\cos(2t) + \frac{5}{4}e^{-t}\sin(2t)} $$

\section*{5.2 非连续强迫函数:单位阶跃函数与平移}

在工程控制系统中,开关的闭合与断开由单位阶跃函数(Unit Step Function,或Heaviside Function)描述:
$$ u(t-a) = \begin{cases} 0, & 0 \le t < a \\ 1, & t \ge a \end{cases} $$
其核心定理为\textbf{第二平移定理}($t$-shifting):
$$ \mathcal{L}\{f(t-a)u(t-a)\} = e^{-as}F(s) \quad \text{以及逆变换} \quad \mathcal{L}^{-1}\{e^{-as}F(s)\} = f(t-a)u(t-a) $$

\subsection*{5.2.1 处理分段定义的微分方程}
处理此类问题的关键是将分段函数改写为阶跃函数形式。

\subsubsection*{典型例题 2}
求解初值问题:
$$ y'' + 4y = f(t), \quad y(0)=0, \, y'(0)=0 $$
其中输入函数 $f(t)$ 定义为:
$$ f(t) = \begin{cases} \sin t, & 0 \le t < \pi \\ 0, & t \ge \pi \end{cases} $$

\textbf{解:}
首先利用阶跃函数重写 $f(t)$:
$$ f(t) = \sin t - \sin t \cdot u(t-\pi) = \sin t + \sin(t-\pi)u(t-\pi) $$
注意:此处我们将 $-\sin t$ 改写为 $+\sin(t-\pi)$,是为了凑出 $f(t-a)$ 的形式以便变换。

对方程两边取拉普拉斯变换:
$$ (s^2 Y(s) - sy(0) - y'(0)) + 4Y(s) = \mathcal{L}\{\sin t\} + \mathcal{L}\{\sin(t-\pi)u(t-\pi)\} $$
代入初值并变换:
$$ (s^2 + 4)Y(s) = \frac{1}{s^2+1} + \frac{e^{-\pi s}}{s^2+1} $$
解出 $Y(s)$:
$$ Y(s) = \frac{1}{(s^2+1)(s^2+4)} (1 + e^{-\pi s}) $$
利用部分分式分解 $\frac{1}{(s^2+1)(s^2+4)} = \frac{1}{3}(\frac{1}{s^2+1} - \frac{1}{s^2+4})$,令 $G(s) = \frac{1}{3}(\frac{1}{s^2+1} - \frac{1}{s^2+4})$,其逆变换为:
$$ g(t) = \frac{1}{3}\left(\sin t - \frac{1}{2}\sin 2t\right) $$
则 $y(t) = \mathcal{L}^{-1}\{G(s)\} + \mathcal{L}^{-1}\{e^{-\pi s}G(s)\}$。利用第二平移定理:
$$ y(t) = g(t) + g(t-\pi)u(t-\pi) $$
具体展开:
$$ y(t) = \frac{1}{3}\left(\sin t - \frac{1}{2}\sin 2t\right) + \frac{1}{3}\left(\sin(t-\pi) - \frac{1}{2}\sin(2(t-\pi))\right)u(t-\pi) $$
化简 $\sin(t-\pi) = -\sin t$ 和 $\sin(2t-2\pi) = \sin 2t$:
$$ \boxed{y(t) = \frac{1}{3}\left(\sin t - \frac{1}{2}\sin 2t\right) - \frac{1}{3}\left(\sin t + \frac{1}{2}\sin 2t\right)u(t-\pi)} $$

\section*{5.3 导数变换性质与高阶方程求解}

拉普拉斯变换最显著的用途在于将微分运算转化为乘法运算。这是求解微分方程(尤其是高阶初值问题)的核心工具。

\subsection*{5.3.1 导数变换定理}
若 $f(t)$ 连续且具有指数阶,则其导数的变换公式为:
$$ \mathcal{L}\{f'(t)\} = sF(s) - f(0) $$
推广到 $n$ 阶导数,公式呈现出明显的规律($s$ 的幂次递减,初值项导数阶数递增):
$$ \mathcal{L}\{f^{(n)}(t)\} = s^n F(s) - s^{n-1}f(0) - s^{n-2}f'(0) - \cdots - f^{(n-1)}(0) $$

\textbf{常用公式速查:}
\begin{itemize}
    \item 二阶导数(考试最高频):
    $$ \mathcal{L}\{y''\} = s^2 Y(s) - s y(0) - y'(0) $$
    \item 三阶导数:
    $$ \mathcal{L}\{y'''\} = s^3 Y(s) - s^2 y(0) - s y'(0) - y''(0) $$
\end{itemize}

\subsection*{5.3.2 高阶初值问题实战}
在处理高阶方程时,务必注意初值 $y(0), y'(0)$ 等前面的负号,以及乘以 $s$ 的幂次。

\subsubsection*{典型例题 5}
求解三阶线性微分方程的初值问题:
$$ y''' + y' = 1, \quad y(0)=0, \, y'(0)=0, \, y''(0)=1 $$

\textbf{解:}
这是一个三阶常系数非齐次方程。
第一步:对方程两边同时取拉普拉斯变换。
利用三阶导数公式:
$$ \mathcal{L}\{y'''\} = s^3 Y(s) - s^2 y(0) - s y'(0) - y''(0) $$
代入题目给定的初值 $y(0)=0, y'(0)=0, y''(0)=1$:
$$ \mathcal{L}\{y'''\} = s^3 Y(s) - 0 - 0 - 1 = s^3 Y(s) - 1 $$
对于一阶导数项:
$$ \mathcal{L}\{y'\} = s Y(s) - y(0) = s Y(s) $$
常数项变换:$\mathcal{L}\{1\} = \frac{1}{s}$。

第二步:代入原方程并整理代数方程。
$$ (s^3 Y(s) - 1) + s Y(s) = \frac{1}{s} $$
合并含有 $Y(s)$ 的项:
$$ (s^3 + s) Y(s) = \frac{1}{s} + 1 = \frac{1+s}{s} $$
$$ s(s^2+1) Y(s) = \frac{s+1}{s} $$
解出 $Y(s)$:
$$ Y(s) = \frac{s+1}{s^2(s^2+1)} $$

第三步:部分分式分解。
观察分母,有二重实根 $s=0$ 和一对复根 $s^2+1=0$。设:
$$ \frac{s+1}{s^2(s^2+1)} = \frac{A}{s} + \frac{B}{s^2} + \frac{Cs+D}{s^2+1} $$
通分比较分子:
$$ s+1 = As(s^2+1) + B(s^2+1) + (Cs+D)s^2 $$
$$ s+1 = (A+C)s^3 + (B+D)s^2 + As + B $$
比较系数:
\begin{itemize}
    \item 常数项:$B = 1$
    \item $s$ 项:$A = 1$
    \item $s^2$ 项:$B + D = 0 \implies D = -1$
    \item $s^3$ 项:$A + C = 0 \implies C = -1$
\end{itemize}
于是:
$$ Y(s) = \frac{1}{s} + \frac{1}{s^2} - \frac{s}{s^2+1} - \frac{1}{s^2+1} $$

第四步:取逆变换得到 $y(t)$。
直接利用基本变换对:
$$ y(t) = \mathcal{L}^{-1}\left\{ \frac{1}{s} \right\} + \mathcal{L}^{-1}\left\{ \frac{1}{s^2} \right\} - \mathcal{L}^{-1}\left\{ \frac{s}{s^2+1} \right\} - \mathcal{L}^{-1}\left\{ \frac{1}{s^2+1} \right\} $$
$$ \boxed{y(t) = 1 + t - \cos t - \sin t} $$

\section*{5.4 广义函数应用:狄拉克 $\delta$ 函数}

当系统受到瞬间的剧烈冲击(如锤击、瞬时高压)时,我们使用狄拉克 $\delta$ 函数(Dirac Delta Function)建模。
$$ \mathcal{L}\{\delta(t-t_0)\} = e^{-st_0}, \quad (t_0 \ge 0) $$
特别地,当 $t_0=0$ 时,$\mathcal{L}\{\delta(t)\} = 1$。这意味着 $\delta(t)$ 的频谱包含所有频率且幅度相等。

\subsection*{5.4.1 含冲击输入的初值问题}
\subsubsection*{典型例题 3}
求解阻尼振动系统在 $t=2$ 时受到冲击的响应:
$$ y'' + 2y' + 2y = \delta(t-2), \quad y(0)=0, \, y'(0)=0 $$

\textbf{解:}
两边取变换:
$$ s^2 Y + 2s Y + 2Y = e^{-2s} $$
$$ Y(s) = \frac{e^{-2s}}{s^2 + 2s + 2} = \frac{e^{-2s}}{(s+1)^2 + 1} $$
不含指数项的部分记为 $H(s) = \frac{1}{(s+1)^2+1}$。
其逆变换为衰减正弦波:
$$ h(t) = \mathcal{L}^{-1}\left\{\frac{1}{(s+1)^2+1}\right\} = e^{-t}\sin t $$
根据含有 $e^{-2s}$ 的形式,应用第二平移定理:
$$ y(t) = h(t-2)u(t-2) $$
$$ \boxed{y(t) = e^{-(t-2)}\sin(t-2) \cdot u(t-2)} $$
\textit{物理诠释:} 系统在 $t<2$ 时保持静止($y=0$),在 $t=2$ 时刻受到冲击,随后开始进行指数衰减的自由振荡。

\section*{5.5 积分方程与卷积定理}

拉普拉斯变换的一个强大性质是将“卷积”转化为“乘积”。
\textbf{卷积定理}:若 $f(t)$ 和 $g(t)$ 的变换分别为 $F(s)$ 和 $G(s)$,则:
$$ \mathcal{L}\{f * g\} = \mathcal{L}\left\{ \int_0^t f(\tau)g(t-\tau) d\tau \right\} = F(s)G(s) $$

\subsection*{5.5.1 求解沃尔泰拉(Volterra)积分方程}
卷积定理常用于求解含有未知函数在积分号内的方程,无需将积分方程转化为微分方程。

\subsubsection*{典型例题 4}
求解未知函数 $f(t)$:
$$ f(t) = t^2 + \int_0^t f(\tau) \sin(t-\tau) d\tau $$

\textbf{解:}
识别右边的积分为卷积形式 $f(t) * \sin t$。原方程可写为:
$$ f(t) = t^2 + f(t) * \sin t $$
两边同时取拉普拉斯变换,利用卷积定理:
$$ F(s) = \frac{2}{s^3} + F(s) \cdot \frac{1}{s^2+1} $$
将含 $F(s)$ 的项移至左边:
$$ F(s) \left( 1 - \frac{1}{s^2+1} \right) = \frac{2}{s^3} $$
通分括号内各项:
$$ F(s) \left( \frac{s^2}{s^2+1} \right) = \frac{2}{s^3} $$
解出 $F(s)$:
$$ F(s) = \frac{2}{s^3} \cdot \frac{s^2+1}{s^2} = \frac{2(s^2+1)}{s^5} = \frac{2}{s^3} + \frac{2}{s^5} $$
取逆变换求解 $f(t)$。已知 $\mathcal{L}^{-1}\{\frac{n!}{s^{n+1}}\} = t^n$:
\begin{itemize}
    \item 对于 $\frac{2}{s^3}$:对应 $2 \cdot \frac{t^2}{2!} = t^2$。
    \item 对于 $\frac{2}{s^5}$:需配凑阶乘,$\frac{2}{4!} \cdot \frac{4!}{s^5} = \frac{1}{12} t^4$。
\end{itemize}
最终结果:
$$ \boxed{f(t) = t^2 + \frac{1}{12}t^4} $$

\chapter*{6 一阶线性方程组}

本章详细阐述常系数一阶线性微分方程组
\begin{equation}
    \mathbf{x}'(t) = A\mathbf{x}(t) + \mathbf{f}(t)
\end{equation}
的四种主要解法。其中 $\mathbf{x}(t)$ 为 $n$ 维列向量,$A$ 为 $n \times n$ 常数矩阵。

\section*{6.1 特征向量法}

\subsection*{6.1.1 原理说明}
当矩阵 $A$ 拥有 $n$ 个线性无关的特征向量 $\mathbf{v}_1, \dots, \mathbf{v}_n$,对应的实特征值为 $\lambda_1, \dots, \lambda_n$ 时,齐次方程组 $\mathbf{x}' = A\mathbf{x}$ 的通解是各特征解的线性组合。

\subsection*{6.1.2 求解步骤}
\begin{enumerate}
    \item 计算特征方程 $\det(A - \lambda I) = 0$,求出特征值 $\lambda_i$。
    \item 对每个 $\lambda_i$,求解 $(A - \lambda_i I)\mathbf{v}_i = \mathbf{0}$ 得到特征向量 $\mathbf{v}_i$。
    \item 若所有特征值互异(或虽有重根但有足够的几何重数),通解形式为:
    \begin{equation}
        \mathbf{x}(t) = c_1 e^{\lambda_1 t}\mathbf{v}_1 + c_2 e^{\lambda_2 t}\mathbf{v}_2 + \dots + c_n e^{\lambda_n t}\mathbf{v}_n
    \end{equation}
\end{enumerate}

\subsection*{6.1.3 完整实例}
\textbf{题目}:求解 $\mathbf{x}' = \begin{pmatrix} 2 & 1 \\ 1 & 2 \end{pmatrix} \mathbf{x}$。

\textbf{解}:
1. \textbf{求特征值}:
   \[
   \det(A - \lambda I) = \begin{vmatrix} 2-\lambda & 1 \\ 1 & 2-\lambda \end{vmatrix} = (\lambda-2)^2 - 1 = \lambda^2 - 4\lambda + 3 = (\lambda-1)(\lambda-3) = 0
   \]
   得到 $\lambda_1 = 1, \lambda_2 = 3$。

2. \textbf{求特征向量}:
   \begin{itemize}
       \item 当 $\lambda_1 = 1$ 时:
       \[
       (A - I)\mathbf{v} = \begin{pmatrix} 1 & 1 \\ 1 & 1 \end{pmatrix} \begin{pmatrix} v_1 \\ v_2 \end{pmatrix} = \mathbf{0} \Rightarrow v_1 + v_2 = 0 \Rightarrow \mathbf{v}_1 = \begin{pmatrix} 1 \\ -1 \end{pmatrix}
       \]
       \item 当 $\lambda_2 = 3$ 时:
       \[
       (A - 3I)\mathbf{v} = \begin{pmatrix} -1 & 1 \\ 1 & -1 \end{pmatrix} \begin{pmatrix} v_1 \\ v_2 \end{pmatrix} = \mathbf{0} \Rightarrow -v_1 + v_2 = 0 \Rightarrow \mathbf{v}_2 = \begin{pmatrix} 1 \\ 1 \end{pmatrix}
       \]
   \end{itemize}

3. \textbf{最终解}:
   \begin{equation}
       \mathbf{x}(t) = c_1 e^{t} \begin{pmatrix} 1 \\ -1 \end{pmatrix} + c_2 e^{3t} \begin{pmatrix} 1 \\ 1 \end{pmatrix}
   \end{equation}

\section*{6.2 广义特征向量法(Jordan 标准型)}

\subsection*{6.2.1 原理说明}
当矩阵 $A$ 存在重复的特征值且其几何重数小于代数重数(即特征向量缺损)时,无法找到 $n$ 个线性无关的特征向量。此时需引入\textbf{广义特征向量}。对于二重特征值 $\lambda$ 只有一个特征向量 $\mathbf{v}$ 的情况,我们需要寻找 $\mathbf{u}$ 满足 $(A-\lambda I)\mathbf{u} = \mathbf{v}$。

\subsection*{6.2.2 求解步骤(以 $2\times 2$ 缺损矩阵为例)}
\begin{enumerate}
    \item 求出重特征值 $\lambda$ 及对应的一个特征向量 $\mathbf{v}$,即 $(A-\lambda I)\mathbf{v} = \mathbf{0}$。
    \item 求解广义特征向量 $\eta$,满足方程 $(A-\lambda I)\eta = \mathbf{v}$。
    \item 构造两个线性无关解:
    \begin{align*}
        \mathbf{x}_1(t) &= e^{\lambda t}\mathbf{v} \\
        \mathbf{x}_2(t) &= e^{\lambda t}(t\mathbf{v} + \eta)
    \end{align*}
    \item 通解为 $\mathbf{x}(t) = c_1 \mathbf{x}_1(t) + c_2 \mathbf{x}_2(t)$。
\end{enumerate}

\subsection*{6.2.3 完整实例}
\textbf{题目}:求解 $\mathbf{x}' = \begin{pmatrix} 3 & 1 \\ 0 & 3 \end{pmatrix} \mathbf{x}$。

\textbf{解}:
1. \textbf{求特征值}:$\det(A-\lambda I) = (3-\lambda)^2 = 0 \Rightarrow \lambda = 3$(代数重数为2)。

2. \textbf{求特征向量}:
   \[
   (A - 3I)\mathbf{v} = \begin{pmatrix} 0 & 1 \\ 0 & 0 \end{pmatrix} \begin{pmatrix} v_1 \\ v_2 \end{pmatrix} = \mathbf{0} \Rightarrow v_2 = 0
   \]
   取 $\mathbf{v} = \begin{pmatrix} 1 \\ 0 \end{pmatrix}$。仅有一个独立特征向量,矩阵缺损。

3. \textbf{求广义特征向量} $\mathbf{u}$:
   解 $(A - 3I)\mathbf{u} = \mathbf{v}$:
   \[
   \begin{pmatrix} 0 & 1 \\ 0 & 0 \end{pmatrix} \begin{pmatrix} u_1 \\ u_2 \end{pmatrix} = \begin{pmatrix} 1 \\ 0 \end{pmatrix} \Rightarrow u_2 = 1
   \]
   $u_1$ 可任意取值,取 $u_1=0$,得 $\mathbf{u} = \begin{pmatrix} 0 \\ 1 \end{pmatrix}$。

4. \textbf{最终解}:
   \[
   \mathbf{x}(t) = c_1 e^{3t} \begin{pmatrix} 1 \\ 0 \end{pmatrix} + c_2 e^{3t} \left[ t \begin{pmatrix} 1 \\ 0 \end{pmatrix} + \begin{pmatrix} 0 \\ 1 \end{pmatrix} \right] = e^{3t} \begin{pmatrix} c_1 + c_2 t \\ c_2 \end{pmatrix}
   \]

\section*{6.3 复数特征值解法}

\subsection*{6.3.1 原理说明}
若矩阵 $A$ 有共轭复特征值 $\lambda = \alpha \pm i\beta$,对应的特征向量也是共轭的 $\mathbf{w} = \mathbf{u} \pm i\mathbf{v}$。利用欧拉公式 $e^{(\alpha + i\beta)t} = e^{\alpha t}(\cos \beta t + i \sin \beta t)$,我们可以提取复数解的实部和虚部作为两个实数基解。

\subsection*{6.3.2 求解步骤}
\begin{enumerate}
    \item 求出特征值 $\lambda = \alpha + i\beta$。
    \item 求出对应的复特征向量 $\mathbf{w} = \mathbf{u} + i\mathbf{v}$。
    \item 构造复数解 $\mathbf{z}(t) = e^{\lambda t}\mathbf{w}$,并展开:
    \[
    \mathbf{z}(t) = e^{\alpha t}(\cos \beta t + i \sin \beta t)(\mathbf{u} + i\mathbf{v})
    \]
    \item 提取实部和虚部作为实值通解的基:
    \begin{align*}
        \mathbf{x}_1(t) &= \text{Re}(\mathbf{z}(t)) = e^{\alpha t}(\mathbf{u} \cos \beta t - \mathbf{v} \sin \beta t) \\
        \mathbf{x}_2(t) &= \text{Im}(\mathbf{z}(t)) = e^{\alpha t}(\mathbf{u} \sin \beta t + \mathbf{v} \cos \beta t)
    \end{align*}
\end{enumerate}

\subsection*{6.3.3 完整实例}
\textbf{题目}:求解 $\mathbf{x}' = \begin{pmatrix} 1 & -1 \\ 1 & 1 \end{pmatrix} \mathbf{x}$。

\textbf{解}:
1. \textbf{求特征值}:
   $\det(A-\lambda I) = (1-\lambda)^2 + 1 = 0 \Rightarrow (1-\lambda)^2 = -1 \Rightarrow \lambda = 1 \pm i$。
   取 $\lambda = 1 + i$,即 $\alpha=1, \beta=1$。

2. \textbf{求特征向量}(针对 $\lambda = 1+i$):
   \[
   (A - (1+i)I)\mathbf{w} = \begin{pmatrix} -i & -1 \\ 1 & -i \end{pmatrix} \begin{pmatrix} w_1 \\ w_2 \end{pmatrix} = \mathbf{0}
   \]
   第一行 $-iw_1 - w_2 = 0 \Rightarrow w_2 = -iw_1$。取 $w_1 = 1$,则 $w_2 = -i$。
   \[
   \mathbf{w} = \begin{pmatrix} 1 \\ -i \end{pmatrix} = \begin{pmatrix} 1 \\ 0 \end{pmatrix} + i \begin{pmatrix} 0 \\ -1 \end{pmatrix} \quad \Rightarrow \quad \mathbf{u} = \begin{pmatrix} 1 \\ 0 \end{pmatrix}, \mathbf{v} = \begin{pmatrix} 0 \\ -1 \end{pmatrix}
   \]

3. \textbf{构造实解}:
   \begin{align*}
       \mathbf{x}_1(t) &= e^t \left[ \begin{pmatrix} 1 \\ 0 \end{pmatrix} \cos t - \begin{pmatrix} 0 \\ -1 \end{pmatrix} \sin t \right] = e^t \begin{pmatrix} \cos t \\ \sin t \end{pmatrix} \\
       \mathbf{x}_2(t) &= e^t \left[ \begin{pmatrix} 1 \\ 0 \end{pmatrix} \sin t + \begin{pmatrix} 0 \\ -1 \end{pmatrix} \cos t \right] = e^t \begin{pmatrix} \sin t \\ -\cos t \end{pmatrix}
   \end{align*}

4. \textbf{最终解}:
   $\mathbf{x}(t) = c_1 e^t \begin{pmatrix} \cos t \\ \sin t \end{pmatrix} + c_2 e^t \begin{pmatrix} \sin t \\ -\cos t \end{pmatrix}$。

\section*{6.4 非齐次方程组解法(待定系数法)}

\subsection*{6.4.1 原理说明}
对于非齐次方程 $\mathbf{x}' = A\mathbf{x} + \mathbf{f}(t)$,若非齐次项 $\mathbf{f}(t)$ 的形式较为简单(如多项式、指数 $e^{\lambda t}$、三角函数 $\sin \omega t, \cos \omega t$),我们可以根据 $\mathbf{f}(t)$ 的形式推测特解 $\mathbf{x}_p(t)$ 的结构,设出待定向量,代入原方程求解系数。

\textbf{设解规则(含共振情况):}
\begin{itemize}
    \item 若 $\mathbf{f}(t) = e^{\mu t}\mathbf{k}$,且 $\mu$ 不是特征值,设 $\mathbf{x}_p = e^{\mu t}\mathbf{a}$。
    \item 若 $\mu$ 是特征值(发生共振),通常需设 $\mathbf{x}_p = e^{\mu t}(\mathbf{a} + \mathbf{b}t)$。
    \item 注意:与标量微分方程不同,即使发生共振,有时也可能不需要 $t$ 项,但设为 $\mathbf{a} + \mathbf{b}t$ 是最通用的安全策略。
\end{itemize}

\subsection*{6.4.2 求解步骤}
\begin{enumerate}
    \item \textbf{求齐次解 $\mathbf{x}_h$}:计算特征值与特征向量,确定齐次通解结构。
    \item \textbf{设定特解形式}:观察 $\mathbf{f}(t)$ 并对比特征值,写出含待定向量 $\mathbf{a}, \mathbf{b}, \dots$ 的表达式 $\mathbf{x}_p(t)$。
    \item \textbf{代入求解}:
    \begin{enumerate}
        \item 计算 $\mathbf{x}_p'(t)$。
        \item 将 $\mathbf{x}_p$ 和 $\mathbf{x}_p'$ 代入方程 $\mathbf{x}' = A\mathbf{x} + \mathbf{f}(t)$。
        \item 比较等式两边同类项(如 $t^k, e^{\lambda t}$ 等)的系数,建立关于 $\mathbf{a}, \mathbf{b}$ 的线性方程组。
        \item 解出待定向量。
    \end{enumerate}
    \item \textbf{组合}:通解 $\mathbf{x}(t) = \mathbf{x}_h(t) + \mathbf{x}_p(t)$。
\end{enumerate}

\subsection*{6.4.3 完整实例}
\textbf{题目}:求解 $\mathbf{x}' = \begin{pmatrix} 2 & 1 \\ 1 & 2 \end{pmatrix} \mathbf{x} + \begin{pmatrix} e^t \\ e^t \end{pmatrix}$。

\textbf{解}:

\textbf{第一步:求齐次通解 $\mathbf{x}_h$}
特征值 $\lambda_1=1, \lambda_2=3$。对应特征向量 $\mathbf{v}_1=(1, -1)^T, \mathbf{v}_2=(1, 1)^T$。
\[
\mathbf{x}_h(t) = c_1 e^t \begin{pmatrix} 1 \\ -1 \end{pmatrix} + c_2 e^{3t} \begin{pmatrix} 1 \\ 1 \end{pmatrix}
\]

\textbf{第二步:设定特解 $\mathbf{x}_p$}
非齐次项 $\mathbf{f}(t) = e^t \begin{pmatrix} 1 \\ 1 \end{pmatrix}$。
指数系数 $\mu = 1$ 与特征值 $\lambda_1 = 1$ 重合(共振)。
为了保险起见,我们设特解形式为一次多项式乘指数:
\[
\mathbf{x}_p(t) = e^t (\mathbf{a} + \mathbf{b}t) = e^t \begin{pmatrix} a_1 \\ a_2 \end{pmatrix} + t e^t \begin{pmatrix} b_1 \\ b_2 \end{pmatrix}
\]
其中 $\mathbf{a}, \mathbf{b}$ 为常向量。

\textbf{第三步:代入方程求解}
1. 对 $\mathbf{x}_p$ 求导:
   \[
   \mathbf{x}_p'(t) = e^t(\mathbf{a} + \mathbf{b}t) + e^t(\mathbf{b}) = e^t(\mathbf{a} + \mathbf{b} + \mathbf{b}t)
   \]
2. 代入原方程 $\mathbf{x}' = A\mathbf{x} + e^t \begin{pmatrix} 1 \\ 1 \end{pmatrix}$:
   \[
   e^t(\mathbf{a} + \mathbf{b} + \mathbf{b}t) = A [e^t(\mathbf{a} + \mathbf{b}t)] + e^t \begin{pmatrix} 1 \\ 1 \end{pmatrix}
   \]
   消去 $e^t$,整理得:
   \[
   \mathbf{b} t + (\mathbf{a} + \mathbf{b}) = A\mathbf{b} t + A\mathbf{a} + \begin{pmatrix} 1 \\ 1 \end{pmatrix}
   \]
3. 比较系数:
   \begin{itemize}
       \item \textbf{$t$ 的系数}:$\mathbf{b} = A\mathbf{b} \Rightarrow (A-I)\mathbf{b} = \mathbf{0}$。
         这意味着 $\mathbf{b}$ 必须是 $\lambda=1$ 的特征向量,即 $\mathbf{b} = k \begin{pmatrix} 1 \\ -1 \end{pmatrix}$。

       \item \textbf{常数项}:$\mathbf{a} + \mathbf{b} = A\mathbf{a} + \begin{pmatrix} 1 \\ 1 \end{pmatrix} \Rightarrow (A-I)\mathbf{a} = \mathbf{b} - \begin{pmatrix} 1 \\ 1 \end{pmatrix}$。
   \end{itemize}

4. 求解 $\mathbf{a}$ 和 $\mathbf{b}$:
   代入 $\mathbf{b}$ 的表达式:
   \[
   \begin{pmatrix} 1 & 1 \\ 1 & 1 \end{pmatrix} \begin{pmatrix} a_1 \\ a_2 \end{pmatrix} = k \begin{pmatrix} 1 \\ -1 \end{pmatrix} - \begin{pmatrix} 1 \\ 1 \end{pmatrix} = \begin{pmatrix} k-1 \\ -k-1 \end{pmatrix}
   \]
   根据线性代数理论,方程 $M\mathbf{x}=\mathbf{y}$ 有解的条件是 $\mathbf{y}$ 在 $M$ 的列空间内。
   矩阵 $\begin{pmatrix} 1 & 1 \\ 1 & 1 \end{pmatrix}$ 的行向量相同,故要求右边向量的上下分量也相同:
   \[
   k-1 = -k-1 \implies 2k = 0 \implies k = 0
   \]
   因此 $\mathbf{b} = \mathbf{0}$。
   
   现在解关于 $\mathbf{a}$ 的方程:
   \[
   \begin{pmatrix} 1 & 1 \\ 1 & 1 \end{pmatrix} \begin{pmatrix} a_1 \\ a_2 \end{pmatrix} = \begin{pmatrix} -1 \\ -1 \end{pmatrix}
   \]
   展开得 $a_1 + a_2 = -1$。这是一个不定方程,我们要找\textbf{一个}特解即可,取最简单的:
   令 $a_1 = -\frac{1}{2}, a_2 = -\frac{1}{2}$。
   
   故特解为:
   \[
   \mathbf{x}_p(t) = e^t \begin{pmatrix} -1/2 \\ -1/2 \end{pmatrix}
   \]

\textbf{第四步:组合通解}
\[
\mathbf{x}(t) = c_1 e^t \begin{pmatrix} 1 \\ -1 \end{pmatrix} + c_2 e^{3t} \begin{pmatrix} 1 \\ 1 \end{pmatrix} - \frac{1}{2}e^t \begin{pmatrix} 1 \\ 1 \end{pmatrix}
\]
\textit{注:由于 $c_1$ 是任意常数,$\mathbf{x}_p$ 中的同类项有时可合并至齐次解中,但在写特解时保留原样即可。}

\end{document}