\documentclass[UTF8,oneside]{ctexbook}

\title{Differential Equations}
\pagestyle{empty}
\date{}
\usepackage{graphicx}
\usepackage{subcaption}
\usepackage{amsmath,amssymb}
\usepackage{enumitem}
\usepackage{tabularx} % 自动调节列宽
\usepackage{booktabs} % 更好看的表格线
\usepackage{array} % 改善列格式


\usepackage[most]{tcolorbox} % 导入tcolorbox

% 定义一个 theorem 风格的方框
\tcbset{
  mytheorem/.style={
    enhanced,
    colback=white,          % 背景颜色
    colframe=red!70!black,  % 边框颜色
    colbacktitle=red!85!black, % 标题背景色
    coltitle=white,         % 标题文字颜色
    fonttitle=\bfseries,    % 标题加粗
    attach boxed title to top left={xshift=2mm,yshift=-2mm}, % 标题位置
    boxed title style={sharp corners}, % 标题方角
    sharp corners,           % 四角直角(不要圆角)
    top=3mm, bottom=3mm, left=3mm, right=3mm, % 内边距
  }
}




\begin{document}


\maketitle

\chapter*{1 一阶微分方程}

\section*{1.1 一阶变量分离性微分方程}

一阶变量分离性(separable)方程的标准形式为:

\[
\frac{dy}{dx} = f(x)g(y)
\]


其中 \(f(x)\) 和 \(g(y)\) 是分别仅依赖于 \(x\) 和 \(y\) 的函数。解此类方程的步骤如下。

一、当$g(y) \ne 0 $时:

\begin{enumerate}
    \item 将变量分离:将所有含 \(y\) 的项移到方程的一侧,含 \(x\) 的项移到另一侧:
\[\frac{1}{g(y)} dy = f(x) dx
\]
    \item 对两边进行积分:
\[\int \frac{1}{g(y)} dy = \int f(x) dx + C
\]

其中 \(C\) 是积分常数。
    \item 得\[ G(y) = F(x) + C \]
\end{enumerate}


二、当$g(y) = 0 $时:

设$y=y_0$是$g(y)=0$的解。

\[
\frac{dy_0}{dx} = f(x)g(y_0)
\]

左右两侧均为$0$,等式成立。因此,$y=y_0$是原方程的一个解。


\section*{1.2 一阶线性微分方程}
一阶线性(linear)微分方程的标准形式为:

\[
\frac{dy}{dx} + P(x)y = Q(x)
\]

其中 \(P(x)\) 和 \(Q(x)\) 是仅依赖于 \(x\) 的已知函数。解此类方程有两个方法。

\subsection*{1.2.1 积分因子法}
积分因子法的本质是将左边变成\textbf{乘积函数}的求导,然后再两边积分。

\begin{enumerate}
    \item 计算积分因子 \(\mu(x)\):
    
\[
\mu (x) = e^{\int P(x) \, dx}
\]
    \item 将原方程两边乘以 $\mu (x)$:
\[
\mu (x) \frac{dy}{dx} + \mu (x) P(x) y = \mu (x) Q(x)
\]

注意到左边是一个乘积函数,则

\[
\frac{d}{dx} [\mu (x) y] = \mu (x) Q(x)
\]
    \item 对两边进行积分:
\[
\mu (x) y = \int \mu (x) Q(x) \, dx + C
\]

\[
\Rightarrow y = \frac{\int \mu(x) Q(x) \, dx + C}{\mu (x)}
\]


\end{enumerate}

\subsection*{1.2.2 常数变易法}
常数变易法的本质是\textbf{非齐次方程的解=齐次方程的通解+一个非齐次方程的特解}。其中,齐次解含有常数$C$,而非齐次特解不含常数$C$。

\begin{enumerate}
    \item 先解对应的齐次方程\[
\frac{dy}{dx} + P(x)y = 0
\]
解得齐次解
\[y = Ce^{-\int P(x) \, dx}\]
    \item 设非齐次解为 $y=\mu (x)e^{-\int P(x) \, dx}$,则
\[
y^{'}=\mu^{'}(x)e^{-\int P(x) \, dx} - \mu (x)P(x)e^{- \int P(x) \, dx}
\]
    \item 将 $y$ 和 $y^{'}$ 代入原方程,化简得  
\[\mu^{'}(x) = Q(x)e^{\int P(x) \, dx}\]
再对$\mu^{'}(x)$积分,得

\[
\mu (x) = \int Q(x)e^{\int P(x) \, dx} \, dx + C
\]

    \item 将$\mu (x)$代入$y=\mu (x)e^{-\int P(x) \, dx}$,得非齐次方程的通解
\end{enumerate}




\subsection*{1.2.3 伯努利(Bernoulli)方程}

伯努利方程(Bernoulli Equations) 是一阶常微分方程

\[
\frac{dy}{dx}+P(X)y=Q(x)y^n
\]

其中,$n \ne 0,1$。方程含非线性项$y^n(n\ne 1)$,可通过变量代换转化为\textbf{一阶线性微分方程求解}。

例题:$y'-\varepsilon y= -\sigma y^{-2}$

令$z=y^{1-n}=y^3$,则$\frac{dz}{dx}=3y^2\frac{dy}{dx}$。左右两边同乘$3y^2$,得

\[
3y^2\frac{dy}{dx}-3\varepsilon y^3=-3\sigma
\]

\[
\Rightarrow \frac{dz}{dx}-3\varepsilon z=-3\sigma
\]

用积分因子法可求解,并将$z=y^3$代回得到结果。

\section*{1.3 零次齐次方程}
零次齐次方程满足:
\[
F(tx,ty) = t^0 F(x,y) = F(x,y) 
\]

\[
\Rightarrow \frac{dy}{dx} = F\left(\frac{y}{x}\right)
\]

右边仅依赖于$\frac{y}{x}$的函数,不含独立常数项。

若右边含有常数项,如
\[
\frac{dy}{dx} = \frac{2x+y+1}{x+2y+3}
\]

可通过变量代换将其化为零次齐次方程。
\begin{itemize}
    \item 令$x=X+h$,$y=Y+k$,其中$h,k$为常数
    \item 令常数项为$0$,即解方程组计算$h,k$,使右端只含$\frac{Y}{X}$的函数
    \item 解零次齐次方程$\frac{dY}{dX}=F\left(\frac{Y}{X}\right)$
    \item 将$X=x-h$,$Y=y-k$代回
    \item 若无法解方程组,则说明该方程不是零次齐次方程
\end{itemize}
 
\vspace{1\baselineskip}
解零次齐次方程的步骤如下:

\begin{enumerate}
    \item 令 $u=\frac{y}{x}$,则 $y=ux$,$\frac{dy}{dx} = u + x\frac{du}{dx}$
    \item 代入$\frac{dy}{dx} = F\left(\frac{y}{x}\right)$,得 
\[
u + x\frac{du}{dx} = F(u)
\]

\[
\frac{du}{F(u) - u} = \frac{dx}{x}
\]

    \item 两边积分
\[
\int \frac{1}{F(u) - u} \, du = \int \frac{1}{x} \, dx
\]

最后将$u=\frac{y}{x}$代回。

\end{enumerate}


\section*{1.4 线性方程与非线性方程}

一阶线性微分方程的标准形式为:
\[\frac{dy}{dx} + P(x)y = Q(x)
\]

其中:
\begin{itemize}
    \item $y$和$\frac{dy}{dx}$均以一次幂出现
    \item 系数只能是关于$x$的函数,不能含有$y$
    \item 不允许出现$y^2$,$y\cdot \frac{dy}{dx}$,$\frac{1}{y}$,$e^y$等非线性项
\end{itemize}

否则,该方程为非线性方程。

\vspace{1\baselineskip}

当方程以$\dfrac{dy}{dx} = f(x,y)$形式出现但求解困难时,可尝试将其整理为

\[
\dfrac{dx}{dy} + P(y)x = Q(y)
\]

交换自变量和因变量。

\subsection*{1.4.1 存在性与唯一性}


\begin{tcolorbox}[mytheorem,
  title=Theorem 4.1 \quad 存在性与唯一性]



无论线性方程or非线性方程,对于初值问题(Initial Value Problem,I.V.P)

\[
\begin{cases}
    y' = f(t,y)\\
    y(t_0)=y_0
\end{cases}
\]

当$f(t,y)$和$\frac{\partial f}{\partial y}$在$(t_0,y_0)$附近连续时,I.V.P存在且唯一解。


\end{tcolorbox}

其中:

\begin{itemize}
    \item $f(t,y)$连续$\rightarrow$I.V.P存在解
    \item $f(t,y)$连续 $\&$ $\frac{\partial f}{\partial y}$连续$\rightarrow$I.V.P存在且唯一解
\end{itemize}


\vspace{2\baselineskip}

\section*{1.5 自治微分方程与非自治微分方程}

\subsection*{1.5.1 自治微分方程}

所有一阶微分方程可分为自治方程和非自治方程。

自治(exact)方程的标准形式为:
\[\frac{dy}{dx} = f(y)
\]

非自治方程的标准形式为:
\[\frac{dy}{dx} = f(x,y)
\]

\subsection*{1.5.2 方向场}
对于一阶微分方程

\[
\frac{dy}{dx} = f(x,y)
\]

在平面上每一点$(x,y)$处,斜率为$f(x,y)$的线段称为该点的方向元素。所有方向元素构成的图形称为该方程的方向场。

\subsubsection{自治方程}


\textbf{平衡解}是$y$与$x$无关的解,即\textbf{常数解}。平衡解满足$\frac{dy}{dx}=0$。

自治方程方向场的图像特征:

\begin{itemize}
    \item 是否含平衡解,取决于$\frac{dy}{dx}=0$是否有解
    \item 若含平衡解$y=y_0$,则在$y=y_0$处方向元素水平
    \item 方向元素与$x$无关,只与$y$有关,表现为每个水平方向上,小斜线均重复
\end{itemize}

\begin{figure}[htbp]
    \centering
    \includegraphics[width=1.0\textwidth]{figures/1.5.2_Autonomous.png}
    \caption{自治方程方向场示例}
    \label{fig:自治方程方向场}
\end{figure}

如图\ref{fig:自治方程方向场}所示,方程$\frac{dp}{dt} = 0.5p -450$的解为$p=900+Ce^{t/2}$,方向场中平衡解$p=900$处方向元素水平。

\subsubsection{非自治方程}

目前遇到的非自治方程的例子——一阶线性非齐次方程

\[
\frac{dy}{dx} + P(x)y = Q(x)
\]

其中 $Q(x) \ne 0$。

解的结构为:齐次解(含$C$)+特解(不含$C$)。

非自治方程方向场的图像特征:

\begin{itemize}
    \item 不含平衡解
    \item 方向元素与$x$有关,每个水平方向上,小斜线不重复
    \item 随着$x \to \pm \infty$,所有解趋向于/远离特解
\end{itemize}

\begin{figure}[htbp]
    \centering
    \includegraphics[width=1.0\textwidth]{figures/1.5.2_Non_Autonomous.png}
    \caption{非自治方程方向场示例}
    \label{fig:非自治方程方向场}
\end{figure}

如图\ref{fig:非自治方程方向场}所示,方程$\frac{dy}{dt} + \frac{1}{2}y = \frac{1}{2}e^{t/3}$的解为$y=\frac{3}{5}e^{t/3}+Ce^{-t/2}$,方向场中无平衡解。

但随着$t \to + \infty$,所有解趋向于特解$y=\frac{3}{5}e^{t/3}$。

\vspace{2\baselineskip}

\subsection*{1.5.3 逻辑斯蒂函数}

逻辑斯蒂函数(Logistic)用于描述具有饱和增长特性的种群增长


\[
\frac{dy}{dt} = r(1-\frac{y}{K})y
\]

其中$r$是增长率,$K$是环境容纳量。解该微分方程,得

\[
y = \frac{K}{1+Ce^{-rt}}
\]

若初始值$y(0)=y_0$,则

\[
y =\frac{y_0 K}{y_o+(K-y_0)e^{-rt}}
\]

\vspace{5\baselineskip}

\subsubsection{平衡解}

令$y'=0$,得平衡解$y=0$和$y=K$。

\begin{itemize}
    \item 当$0<y<K$,$\frac{dy}{dt}>0$,$y(t)$单调递增
    \item 当$y>K$,$\frac{dy}{dt}<0$,$y(t)$单调递减
\end{itemize}



$\therefore$  当$ t \rightarrow \infty,~ y = \frac{K}{1+Ce^{-rt}} \rightarrow K$


即解趋近于$K$。

\vspace{1\baselineskip}

稳定的平衡解指:系统暂时偏离这个平衡解,但最终仍能回到平衡解。

\begin{itemize}
    \item 对于$y=0$:当$y$略大于$0$时,$\frac{dy}{dt} > 0$,即$y$远离平y衡解$0$。$\therefore y=0$不是稳定的平衡解。
    \item 对于$y=K$:当$y$略大于$K$时,$\frac{dy}{dt} < 0$,即$y$靠近平衡解$K$;当$y$略小于$K$时,$\frac{dy}{dt} > 0$,即$y$靠近平衡解$K$。$\therefore y=K$是稳定的平衡解。
\end{itemize}




\subsubsection{凹凸性与拐点}

计算$y$的二阶导

\[
\frac{d^2y}{dt^2} = r^2(1-\frac{y}{K})y(1-\frac{2y}{K})
\]

随着$t \rightarrow\infty$,最终$y\rightarrow \infty$。故$0<y<K$。

\begin{itemize}
    \item 当$0<y<\frac{K}{2},\frac{d^y}{dt^2} > 0$,曲线是凹的。
    \item 当$\frac{K}{2}<y<K,\frac{d^y}{dt^2} < 0$,曲线是凸的。
\end{itemize}

拐点处$\frac{d^y}{dt^2}=0$,故拐点为$y=\frac{K}{2}$。

\vspace{7\baselineskip}

\section*{1.6 恰当方程与积分因子}


\subsection*{1.6.1 恰当方程}
\begin{tcolorbox}[mytheorem,
    title=Theorem 1.6.1 \quad 恰当性的判定定理]

对于方程

\[
M(x,y) \, dx+N(x,y) \, dy=0
\]
    
若存在二元函数$u(x,y)$,使$\frac{\partial u}{\partial x}=M(x,y),\frac{\partial u}{\partial y}=N(x,y)$,则方程$M \, dx+N \, dy=0$称为恰当方程。
且充要条件为

\[
\frac{\partial M}{\partial y}=\frac{\partial N}{\partial x}
\]

\end{tcolorbox}

例题:$(2x+y) \, dx+(x+2y) \, dy=0$

设$M=2x+y,N=x+2y.$则 $\frac{\partial M}{\partial y}=1=\frac{\partial N}{\partial x}$,符合恰当方程的定义。




\[
\frac{\partial u}{\partial x}=2x+y  \Rightarrow u = \int (2x+y) \, dx = x^2+xy+ \underset{\text{常数可能与 $y$ 有关}}{\varphi(y)}
\]

\[
\Rightarrow \frac{\partial u}{\partial y}=x+\varphi ' (y) =N = x+2y
\]

$\therefore \varphi '(y)=2y,\varphi(y)=y^2+C$。

$\therefore u(x,y)=x^2+xy+y^2$,通解为$x^2+xy+y^2=C$。

\subsection*{1.6.2 积分因子}

设积分因子$v(x,y)$,使$v(x,y)M \, dx+v(x,y) N \, dy =0$为恰当方程。
则需满足:

\[
\frac{\partial(vM)}{\partial y} = \frac{\partial (vN)}{\partial x} \Rightarrow Mv_y-Nv_x+(M_y-N_x) v=0
\]

\begin{itemize}
    \item 当$v$仅依赖$x$,则满足$\frac{dv}{dx}=\frac{M_y-N_x}{N}v$,此时$v=v(x)$
    \item 当$v$仅依赖$y$,则满足$\frac{dv}{dy}=\frac{M_y-N_x}{-M}v$,此时$v=v(y)$
\end{itemize}

\section*{1.7 数值逼近:欧拉方法}

欧拉方法是求解一阶初值问题的基础数值近似方法。

\[
\begin{cases}
    \text{D.E.} \quad \dfrac{dy}{dt} = f(t,y) \\
    \text{I.C.} \quad y(0) = y_0
\end{cases}
\]

\paragraph{1. 局部切线近似:}

\quad

解函数$y=\phi(t)$在$t=t_0$处的切线斜率为$\dfrac{dy}{dt} = f(t_0,y_0)$,在$t_0$附近,用切线方程近似解曲线

\[
y_1 = y_0 + f(t_0,y_0)(t_1-t_0)
\]

\paragraph{2. 迭代扩展}

取步长$h = t_{n+1} - t_n$(即每一步的间隔相同),得迭代公式为

\[
y_{n+1} = y_n + h \cdot f(t_n,y_n), \quad n = 0, 1, 2, \cdots
\]

欧拉方法易实现,但精度有限:

\begin{itemize}
    \item 步长$h$越小,精度越高,但计算量越大
    \item 收敛解族的误差增长缓慢,但发散解族的误差会急剧放大,需采用更高效的数值方法
\end{itemize}

\vspace{14\baselineskip}


\section*{1.8 解的存在唯一性定理}

\subsection*{1.8.1 解的存在唯一性定理}

\begin{tcolorbox}[mytheorem,
    title=Theorem 1.8.1 \quad 一阶微分方程的存在唯一性定理]

考虑初值问题

\[
\begin{cases}
    y' = f(t,y)\\
    y(0) = 0
\end{cases}
\]
    
若f(t,y)在矩形区域$R:|t| \le a,|y| \le b$上连续,则该初值问题在该区域内存在唯一解。

\end{tcolorbox}

\subsection*{1.8.2 微分方程与积分方程的等价性}

对于初值问题

\[
\begin{cases}
    y' = f(t,y)\\
    y(t_0) = y_0
\end{cases}
\]

对$y'=f(t,y)$从$t_0$到$t$积分,并结合初始条件$y(t_0) = y_0$,可得积分方程

\[
\phi (t) = y_0 + \int_{t_0}^{t}f(s,\phi(s)) \, ds
\]

即:微分方程初值问题的解$\phi(t)$与该积分方程的解完全等价。

\subsection*{1.8.3 皮卡逐步逼近法}

对于初值问题

\[
\begin{cases}
    y' = f(t,y)\\
    y(t_0) = y_0
\end{cases}
\]

定义常数函数$\phi_0(t_0) = y_0$,有第n次近似解

\[
\phi_{n+1}(t) = y_0 + \int_{t_0}^t f(s,\phi_n(s)) \, ds
\]

\subsection*{1.8.4 里卡蒂(Riccati)方程}

Riccati方程是一阶非线性常微分方程,其标准形式为

\[
\dfrac{dy}{dt} = q_1(t)+q_2(t)y+q_3(t)y^2
\]

需已知某特解$y_1(t)$,令$y(t) = z(t) + y_1(t)$,则

\[
\dfrac{dy}{dt} = \frac{dz}{dt} + \frac{dy_1}{dt}
\]

将其代入原方程

\[
\begin{aligned}
    \frac{dz}{dt}+\frac{dy_1}{dt}
    &= q_1(t)+q_2(t)(z+y_1)+q_3(t)(z+y_1)^2 \\
    &= \left[q_1(t)+q_2(t)y_1+q_3(t)y_1^2\right]
    + q_2(t)z + q_3(t)z^2 + 2q_3(t)y_1 z
\end{aligned}
\]

由于$\dfrac{dy}{dt} = q_1(t)+q_2(t)y_1+q_3(t)y_1^2$,上式可简化

\[
\frac{dz}{dt} = (q_2(t) + 2q_3(t)y_1(t))z+q_3(t)z^2
\]

化简为Bernoulli方程,作变量替换$u=z^{-1}$可解。

\chapter*{2 二阶微分方程}

\section*{2.1 可降阶的二阶微分方程}

降阶法科求解部分\text{非线性}、\text{变系数}的二阶方程。

\subsection*{2.1.1 $y''=f(x)$}

仅含$x$。两次积分即可求解。

\begin{itemize}
    \item 第一次积分:$y'=\int f(x) \, dx + C_1$
    \item 第二次积分:$y=\int (\int f(x) \, dx + C_1) \, dx +C_2$
\end{itemize}

\subsection*{2.1.2 $y''=f(x,y')$}

不含$y$。令$p=y'$,则$y''=\dfrac{dp}{dx}$,方程变为一阶方程$\dfrac{dp}{dx}f(x,p)$。

\subsection*{2.1.3 $y''=f(y,y')$}

不含$x$。令$p=y'$,则$y''=p\dfrac{dp}{dy}$,方程变为一阶方程$p\dfrac{dp}{dy}=f(y,p)$。

\vspace{5\baselineskip}

\section*{2.2 二阶线性常系数微分方程}

对于二阶线性常系数齐次方程,有以下定理

\[
y'' + py' + qy = 0
\]

\subsubsection*{1. 微分算子与存在唯一性定理}

\paragraph{微分算子}

\quad

设$p(x),q(x)$是区间$I$上的连续函数,定义微分算子

\[
L[\phi] = \phi''+p(x)\phi'+q(x)\phi
\]

它将函数$\phi_(x)$映射为另一个函数。

\paragraph{存在唯一性定理}

\quad

对于初值问题

\[
\begin{cases}
    y''+p(x)y'+q(t)y = f(x)\\
    y(x_0) = y_0,y'(x_0) = y_0'
\end{cases}
\]

若$p,q,f$在包含$t_0$的开区间$I$上连续,则该问题在$I$上存在唯一解。

\subsubsection*{2. 叠加原理}

若$y_1,y_2$是$L[y] = 0$的解,则对任意常数$c_1,c_2$,线性组合$y=c_1y_1+c_2y_2$也是解。

\subsubsection*{3. 朗斯基行列式(Wronskian)与基本解组}

对两个解$y_1,y_2$,定义朗斯基行列式

\[
W[y_1,y_2](t) = 
\begin{vmatrix}
y_1(t)&y_2(t)\\
y_1'(t)&y_2(t)
\end{vmatrix}
\]

若$W[y_1,y_2](t)\ne 0$,则$y_1,y_2$线性无关,构成基本解组,方程的通解为

\[
y=c_1y_1+c_2y_2
\]

\subsubsection*{4. 复值解的实部与虚部}

若$y=u(t)+iv(t)$是方程$L[y]=0$的复值解,则其实部$u(t)$和虚部$v(t)$也是方程的解。

\subsubsection*{5. 阿贝尔定理(Abel's Theorem)}

若$y_1,y_2$是方程$y''+p(t)y'+q(t)y=0$的解,则朗斯基行列式满足

\[
W[y_1,y_2](t) = ce^{-\int p(t) \, dt} 
\]

其中$c=W(t_0)$

\subsubsection*{6. 克莱姆法则(Cramer)}

设二阶线性齐次微分方程的通解为$y(t) = c_1y_1(t) +c_2y_2(t)$,给定初值条件$y(t_0)=y_0,y'(t_0)=y_0'$,需要确定习俗$c_1,c_2$。将初值代入通解

\[
\begin{cases}
    c_1y_1(t_0)+c_2y_2(t_0)=y_0\\
    c_1y_1'(t_0)+c_2y_2'(t_0)=y_0'
\end{cases}
\]

\paragraph{求$c_1$:}

\quad

将系数矩阵的第一列替换为初值向量
$\begin{bmatrix}
    y_0\\y_0'
\end{bmatrix}$,则

\[
c_1=\dfrac{\begin{vmatrix}y_0&y_2(t_0)\\y_0'&y_2'(t_0)\end{vmatrix}}{W[y_1,y_2](t_0)}
\]

\paragraph{求$c_2$:}

\quad

将系数矩阵的第二列替换为初值向量
$\begin{bmatrix}
    y_0\\y_0'
\end{bmatrix}$,则

\[
c_2=\dfrac{\begin{vmatrix}y_1(t_0)&y_0\\y_1'(t_0)&y_0'\end{vmatrix}}{W[y_1,y_2](t_0)}
\]

利用朗斯基行列式判断基本解组的线性无关性,再用克莱姆法则,在已知初值条件时确定通解中的系数$c_1,c_2$。

\subsection*{恰当方程与伴随方程}


\subsection*{2.2.1 齐次方程}

考虑齐次方程

\[
y''+py'+qy=0
\]

\begin{enumerate}[label=STEP\arabic*]
    \item 写特征方程:$r^2+pr+q=0$
    \item 解特征根,按根的类型定通解:
        \begin{itemize}
            \item $r_1 \ne r_2$:$y=C_1e^{r_1x} + C_2e^{r_2x}$
            \item $r_1=r_2=r$:$=C_1e^{rx}+C_2xe^{rx}$
            \item 共轭复根$r=\alpha \pm i \beta$:$y=e^{\alpha x}(C_1 \cos \beta x + C_2 \sin \beta x)$
        \end{itemize}
\end{enumerate}

\subsection*{2.2.2 非齐次方程}

考虑非齐次方程

\[
y''+py'+qy=f(x) \; (f(x)\not\equiv 0)
\]

通解=齐次通解+非齐次通解。

用待定系数法设特解形式,该方法仅适用于$f(x)$是多项式、指数函数、正余弦函数,或它们的乘积。

\subsubsection*{1. $f(x)=P_m(x)$($m$次多项式,如$f(x)=3x^2+2x+1$)}

\begin{table}[htbp]
\centering
\begin{tabular}{c c}
\toprule
特征根与$r=0$的关系 & 特解$y_p$假设形式  \\
\midrule
$r=0$不是特征根 & $y_p=Q_m(x)=Ax^2+Bx+C$  \\
$r=0$是单特征根 & $y_p=x\cdot Q_m(x)$  \\
$r=0$是二重特征根 & $y_p=x^2\cdot Q_m(x)$  \\
\bottomrule
\end{tabular}
\end{table}

\vspace{4\baselineskip}

\subsubsection*{2. $f(x)=Ae^{\alpha x}$(或$Pn(x)e^{\alpha x}$,多项式×指数的本质一致}

\begin{table}[htbp]
\centering
\begin{tabular}{c c}
\toprule
特征根与$\alpha$的关系 & 特解$y_p$假设形式  \\
\midrule
$\alpha$不是特征根 & $y_p=Be^{\alpha x}$  \\
$\alpha$是单特征根 & $y_p=x\cdot Be^{\alpha x}$  \\
$\alpha$是二重特征根 & $y_p=x^2\cdot Be^{\alpha x}$  \\
\bottomrule
\end{tabular}
\end{table}

\subsubsection*{3. $f(x) = A\cos (\beta x)+B\sin (\beta x)$(或$P_n(x)[A\cos (\beta x)+B\sin (\beta x)]$,多项式×三角函数的本质一致}

\begin{table}[htbp]
\centering
\begin{tabular}{c c}
\toprule
特征根与$i\beta$的关系 & 特解$y_p$假设形式  \\
\midrule
$i\beta$不是特征根 & $y_p=C\cos (\beta x)+D\sin (\beta x)$  \\
$i\beta$是特征根 & $y_p=x\cdot C\cos (\beta x)+D\sin (\beta x)$  \\
\bottomrule
\end{tabular}
\end{table}

\subsubsection*{4. $f(x)=e^{\alpha x}(A\cos (\beta x)+B\sin (\beta x))$(指数×正余弦)}

\begin{table}[htbp]
\centering
\begin{tabular}{c c}
\toprule
特征根与$\alpha \pm i\beta$的关系 & 特解$y_p$假设形式  \\
\midrule
$\alpha \pm i\beta$不是特征根 & $y_p=e^{\alpha x}(C\cos (\beta x)+D\sin (\beta x))$  \\
$\alpha \pm i\beta$是特征根 & $y_p=x\cdot e^{\alpha x}(C\cos (\beta x)+D\sin (\beta x))$  \\
\bottomrule
\end{tabular}
\end{table}


\section*{2.3 二阶线性变系数微分方程}

欧拉方程是典型的二阶线性变系数微分方程

\[
t^2y''(t)+\alpha t y'(t)+\beta y(t) = 0
\]

设$t=e^x$,将$y$对$t$的导数转化为对$x$的导数

\[
\dfrac{d^2y}{dx^2}+(\alpha-1)\dfrac{dy}{dx}+\beta y = 0
\]




\chapter*{3 高阶微分方程}

\section*{3.1 高阶微分方程解的理论}





\section*{3.2 常系数齐次线性微分方程}

对n阶方程$L[y]=a_0y^{(n)}+a_1y^{(n-1)}+\cdots+a_n y=0$,假设解为$y=e^{rt}$,代入得特征多项式

\[
Z(r) = a_0r^n+a_1r^{n-1}+\cdots+a_n
\]

\subsection*{1. 实根且互不相等}

若有n个不同实根$r_1,r_2,\cdots,r_n$,则通解为

\[
y=c_1e^{r_1t}+c_2e^{r_2t}+\cdots+c_ne^{r_n t}
\]

\subsection*{2. 复共轭根}

若有复共轭根$r=\lambda\pm i \mu$。则对应是实值解为$e^{\lambda t}\cos \mu t,e^{\lambda t} \sin \mu t$。

\subsection*{3. 重根}

若$r$是k重实根,对应解为$e^{rt},te^{rt},\cdots,t^{k-1}e^{rt}$。

\section*{3.3 待定系数法}




\section*{3.4 常数变易法}

设齐次方程的n个线性无关解为$y_1,y_2,\cdots,y_n$,则齐次通解为

\[
y_h=c_1y_1+c_2y_2+\cdots+c_ny_n
\]

将常数$c_1,\cdots,c_n$替换为待定函数$u_1(t),\cdots,u_n(t)$,假设特解

\[
y_p=u_1y_1+u_2y_2+\cdots+u_ny_n
\]

\[
W=
\begin{vmatrix}
    y_1&y_2&\cdots&y_n\\
    y_1'&y_2'&\cdots&y_n'\\
    \vdots&\vdots&\ddots&\vdots\\
    y_1^{(n-1)}&y_2^{(n-1)}&\cdots&y_n^{(n-1)}
\end{vmatrix}
\]

替换行列式$W_m$:将$W$的第$m$列替换为
$\begin{bmatrix}
    0&0&\cdots&g(t)
\end{bmatrix}^T$(共n行)

则$u_m'=\dfrac{W_m}{W}$。

\end{document}